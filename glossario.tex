
%**************************************************************
% Acronimi
%**************************************************************
%\renewcommand{\acronymname}{Acronimi e abbreviazioni}
%
%\newacronym[description={\glslink{apig}{Application Program Interface}}]
%    {api}{API}{Application Program Interface}
%
%\newacronym[description={\glslink{umlg}{Unified Modeling Language}}]
%    {uml}{UML}{Unified Modeling Language}

%**************************************************************
% Glossario
%**************************************************************
%\renewcommand{\glossaryname}{Glossario}

%\newglossaryentry{ajax}
%{
%    name=AJAX,
%    text=Asynchronous JavaScript and XML,
%    sort=ajax,
%    description={Asynchronous JavaScript and XML è una tecnica di sviluppo software per la realizzazione di applicazioni web che interagiscano in background con il server senza bisogno di ricaricare la pagina nel browser}
%}

\newglossaryentry{back-end}
{
    name=Back-end,
    text=Back-end,
    sort=Back-end,
    description={Denotazione della parte del software invisibile all'utente, ovvero quella che permette l'effettivo funzionamento del sistema}
}

\newglossaryentry{batch}
{
    name=BATCH,
    text=BATCH,
    sort=batch,
    description={L'esecuzione non immediata, ma rimandata nel tempo di programmi}
}

\newglossaryentry{ced}
{
    name=CED,
    text=Centro Elaborazione Dati,
    sort=ced,
    description={Un centro elaborazione dati (CED) è l'unità organizzativa che coordina e mantiene le apparecchiature ed i servizi di gestione dei dati, ovvero l'infrastruttura IT a servizio di una o più aziende}
}

\newglossaryentry{cics}
{
    name=CICS,
    text=Customer Information Control System,
    sort=cics,
    description={Customer Information Control System è una famiglia di application server che fornisce la gestione di transazioni online e connettività per applicazioni su mainframe IBM}
}

\newglossaryentry{cobol}
{
    name=COBOL,
    text=COmmon Business-Oriented Language,
    sort=cobol,
    description={COmmon Business-Oriented Language è uno dei primi linguaggi di programmazione ad essere stato sviluppato. Nonostante sia un linguaggio datato, il COBOL è tuttora presente in molte applicazioni software commerciali di tipo bancario, specie lato mainframe (es. CICS), che non si è preferito o voluto migrare in altra tecnologia software}
}

%\newglossaryentry{covenant}
%{
%    name=Covenant,
%    text=Covenant,
%    sort=covenant,
%    description={In finanza con il termine covenant si indica un accordo che intercorre tra un'impresa e i suoi finanziatori, includendo determinate clausole e parametri per tutelarli}
%}

%\newglossaryentry{ctg}
%{
%    name=CTG,
%    text=CICS Transaction Gateway,
%    sort=ctg,
%    description={CICS Transaction Gateway offre un accesso sicuro a sistemi CICS da applicazioni Java, Java EE, .NET, C e C++ usando i protocolli internet}
%}
 
\newglossaryentry{dbms}
{
    name=DBMS,
    text=Database Management System,
    sort=dbms,
    description={Database Management System è un sistema di gestione di basi di dati, consente la creazione, la manipolazione e l'interrogazione efficiente di database}
}

\newglossaryentry{eis}
{
    name=EIS,
    text=Enterprise Information Systems,
    sort=eis,
    description={Enterprise Information Systems, sistemi informativi d'impresa per migliorare le funzionalità dei processi di business delle aziende rapportandosi con grandi quantità di dati e instaurando un sistema di gestione centrale delle informazioni}
}

\newglossaryentry{elise}
{
    name=ELISE,
    text=Extended Loans Integrated SystEm,
    sort=elise,
    description={Extended Loans Integrated SystEm è un sistema per la gestione integrata di tutte le problematiche di business relative all'area dei finanziamenti}
}

\newglossaryentry{erp}
{
    name=ERP,
    text=Enterprise Resource Planning,
    sort=erp,
    description={Enterprise Resource Planning, letteralmente "pianificazione delle risorse d'impresa", è un sistema di gestione, chiamato in informatica "sistema informativo", il quale integra tutti i processi di business rilevanti di un'azienda (vendite, acquisti, gestione magazzino, contabilità ecc.)}
}


\newglossaryentry{front-end}
{
    name=Front-end,
    text=Front-end,
    sort=Front-end,
    description={Denotazione della parte del software visibile all'utente, spesso un'interfaccia grafica con cui egli può interagire ed usufruire delle funzionalità offerte}
}


%\newglossaryentry{iframe}
%{
%    name=iframe,
%    text=iframe,
%    sort=iframe,
%    description={inline frame è usato per l'inclusione di una finestra su un'altra risorsa nella pagina corrente}
%}

\newglossaryentry{ict}
{
    name=ICT,
    text=Information and Communications Technology,
    sort=ict,
    description={Information and Communications Technology, ovvero le tecnologie dell’informazione e della comunicazione, spesso questo acronimo è usato per definire l'ambito di esercizio delle attività di un'azienda}
}

\newglossaryentry{ispf}
{
    name=ISPF,
    text=Interactive System Productivity Facility,
    sort=ispf,
    description={ISPF è un software per i sistemi operativi z/OS che vengono eseguiti sui mainframe IBM. ISPF inoltre è un User-Interface Development Environment, ovvero un ambiente di sviluppo con interfaccia grafica}
}

%\newglossaryentry{jar}
%{
%    name=JAR,
%    text=Java Archive,
%    sort=jar,
%    description={Java Archive è un archivio che raccoglie in un unico file le classi Java di un programma e le relative risorse, per distribuire il software in una piattaforma compatibile}
%}

%\newglossaryentry{javabeans}
%{
%    name=JavaBeans,
%    text=JavaBeans,
%    sort=javabeans,
%    description={è una particolare convenzione per scrivere classi Java in modo da includere più oggetti in uno solo, permettendo la serializzazione e l'interscambio dell'intero insieme o l'impostazione e la fruizione dei singoli attributi}
%}

%\newglossaryentry{jca}
%{
%    name=JCA,
%    text=Java EE Connector Architecture,
%    sort=jca,
%    description={Java EE Connector Architecture è una tecnologia basata su linguaggio Java per la connessione di application server e EIS come parte dell'applicazione d'impresa}
%}

\newglossaryentry{jcl}
{
    name=JCL,
    text=Job Control Language,
    sort=jcl,
    description={In informatica il Job Control Language (JCL) è un linguaggio di scripting utilizzato nei sistemi operativi IBM DOS/VSE, OS/VS1 ed MVS per eseguire (in gergo lanciare) una procedura batch su un sistema generalmente mainframe}
}

%\newglossaryentry{jdbc}
%{
%    name=JDBC,
%    text=Java DataBase Connectivity,
%    sort=jdbc,
%    description={Java DataBase Connectivity è una tecnologia usata per connettere applicazioni Java EE ai diversi database per l'accesso e la gestione della persistenza dei dati}
%}


%\newglossaryentry{jndi}
%{
%    name=JNDI,
%    text=Java Naming and Directory Interface,
%    sort=jndi,
%    description={Java Naming and Directory Interface è un servizio offerto da Java per ottenere dati e oggetti tramite un nome, che possono essere memorizzati su un server, su file o su database}
%}

%\newglossaryentry{mvc}
%{
%    name=MVC,
%    text=Model View Controller,
%    sort=mvc,
%    description={Model View Controller, un pattern architetturale molto diffuso nello sviluppo di sistemi software, in particolare nell’ambito della programmazione ad oggetti}
%}

\newglossaryentry{qmf}
{
    name=QMF,
    text=Query Management Facility,
    sort=qmf,
    description={IBM DB2 Query Management Facility fornisce una serie di funzioni per analytics, dashboard, reporting e query IBM DB2. I report grafici e le visualizzazioni dei dati integrati estendono i vantaggi del prodotto dagli utenti tecnici a una comunità più vasta di utenti di business}
}


\newglossaryentry{rdbms}
{
    name=RDBMS,
    text=Relational Database Management System,
    sort=rdbms,
    description={Relational Database Management System è un sistema per la gestione di basi di dati basato sul modello relazionale}
}


%\newglossaryentry{reflection}
%{
%    name=Reflection,
%    text=Reflection,
%    sort=reflection,
%    description={è una tecnica informatica con cui un programma puo esaminare e modificare la propria struttura e il suo comportamento durante l'esecuzione}
%}

%\newglossaryentry{servlet}
%{
%    name=Servlet,
%    text=Servlet,
%    sort=servlet,
%    description={è un particolare tipo di classe Java fornito dalla versione Java EE per estendere le funzionalità di un server, implementando nelle applicazioni web la controparte	di altre tecnologie per contenuti web dinamici}
%}