% I seguenti commenti speciali impostano:
% 1. 
% 2. PDFLaTeX come motore di composizione;
% 3. tesi.tex come documento principale;
% 4. il controllo ortografico italiano per l'editor.

% !TEX encoding = UTF-8
% !TEX TS-program = pdflatex
% !TEX root = Lahmer_Abdelilah_tesi.tex
% !TEX spellcheck = it-IT

\documentclass[11pt,                    % corpo del font principale
               a4paper,                 % carta A4
               twoside,                 % impagina per fronte-retro
               openright,               % inizio capitoli a destra
               english,                 
               italian,                 
               ]{book}    

\usepackage[utf8]{inputenc}             % codifica di input; anche [latin1] va bene
                                        % NOTA BENE! va accordata con le preferenze dell'editor

%**************************************************************
% Importazione package
%************************************************************** 

%\usepackage{amsmath,amssymb,amsthm}    % matematica

\usepackage[english, italian]{babel}    % per scrivere in italiano e in inglese;
                                        % l'ultima lingua (l'italiano) risulta predefinita

\usepackage{bookmark}                   % segnalibri

\usepackage{caption}                    % didascalie

\usepackage{chngpage,calc}              % centra il frontespizio

\usepackage{csquotes}                   % gestisce automaticamente i caratteri (")

\usepackage{emptypage}                  % pagine vuote senza testatina e piede di pagina

\usepackage{epigraph}					% per epigrafi

\usepackage{eurosym}                    % simbolo dell'euro

\usepackage[T1]{fontenc}                % codifica dei font:
                                        % NOTA BENE! richiede una distribuzione *completa* di LaTeX

%\usepackage{indentfirst}               % rientra il primo paragrafo di ogni sezione

\usepackage{graphicx}                   % immagini

\usepackage{float} 						% per il floated delle immagini

\usepackage{hyperref}                   % collegamenti ipertestuali

\usepackage[binding=5mm]{layaureo}      % margini ottimizzati per l'A4; rilegatura di 5 mm

\usepackage{listings}                   % codici

\usepackage{microtype}                  % microtipografia

\usepackage{mparhack,fixltx2e,relsize}  % finezze tipografiche

\usepackage{nameref}                    % visualizza nome dei riferimenti                                      

\usepackage[font=small]{quoting}        % citazioni

%\usepackage{subfig}                     % sottofigure, sottotabelle

\usepackage{subfigure}                     % sottofigure, sottotabelle

\usepackage{floatflt}

\usepackage[italian]{varioref}          % riferimenti completi della pagina

\usepackage[dvipsnames, table]{xcolor}         % colori

\usepackage{booktabs}                   % tabelle                                       
\usepackage{tabularx}                   % tabelle di larghezza prefissata                                    
\usepackage{longtable}                  % tabelle su più pagine                                        
\usepackage{ltxtable}                   % tabelle su più pagine e adattabili in larghezza

\usepackage[toc,section=chapter,numberedsection=autolabel,nonumberlist]{glossaries}   			% glossario
                                        % per includerlo nel documento bisogna:
                                        % 1. compilare una prima volta tesi.tex;
                                        % 2. eseguire: makeindex -s tesi.ist -t tesi.glg -o tesi.gls tesi.glo
                                        % 3. eseguire: makeindex -s tesi.ist -t tesi.alg -o tesi.acr tesi.acn
                                        % 4. compilare due volte tesi.tex.

\usepackage[backend=biber,style=verbose-ibid,hyperref,backref]{biblatex}
                                        % eccellente pacchetto per la bibliografia; 
                                        % produce uno stile di citazione autore-anno; 
                                        % lo stile "numeric-comp" produce riferimenti numerici
                                        % per includerlo nel documento bisogna:
                                        % 1. compilare una prima volta tesi.tex;
                                        % 2. eseguire: biber tesi
                                        % 3. compilare ancora tesi.tex.
                                        
\usepackage[Lenny]{fncychap}            % intestazione capitoli

\usepackage{fancyhdr}					% modifica header e footer

\usepackage{geometry}
										% modifica i bordi della pagina

%imposta livelli di dettaglio indice
\setcounter{tocdepth}{3}
\setcounter{secnumdepth}{3}

%interlinea fisso
%\linespread{1}

%interlinea itemize fisso
	%\usepackage{enumitem}
	%\setlist[itemize]{noitemsep}
		%\usepackage{enumitem}
		%\setlist[enumerate]{nosep}
		%\setlist[itemize]{nosep}
		%\setlist[description]{nosep}

%mette bottom più alto
%\raggedbottom
%\flushbottom

%per l'uso del simbolo °
\usepackage{siunitx}

%per l'uso dei commenti
\usepackage{verbatim} 

\input{tesi-config}                     % file con le impostazioni personali

\begin{document}

%**************************************************************
% Materiale iniziale
%**************************************************************

\frontmatter
\pagenumbering{Roman}
% !TEX encoding = UTF-8
% !TEX TS-program = pdflatex
% !TEX root = ../tesi.tex
% !TEX spellcheck = it-IT

\blankpage
\blankpage
%**************************************************************
% Frontespizio 
%**************************************************************
\begin{titlepage}

\begin{center}

\begin{LARGE}
\textbf{\myUni}\\
\end{LARGE}

\vspace{12.5pt}

\begin{Large}
\textsc{\myDepartment}\\
\end{Large}

\vspace{12.5pt}

\begin{large}
\textsc{\myFaculty}\\
\end{large}

\vspace{30pt}
\begin{figure}[htbp]
\begin{center}
\includegraphics[height=6cm]{logo-unipd}
\end{center}
\end{figure}
\vspace{30pt} 

\begin{LARGE}
\begin{center}
\textbf{\myTitle}\\
\end{center}
\end{LARGE}

\vspace{10pt} 

\begin{large}
\textsl{\myDegree}\\
\end{large}

\vspace{40pt} 

%\begin{large}
%\begin{flushleft}
%\textit{Laureando}\\ 
%\vspace{5pt} 
%\myName
%\end{flushleft}
%
%%\vspace{0pt}
%\hfill
%
%\begin{flushright}
%\textit{Relatore}\\ 
%\vspace{5pt} 
%Prof. \myProf
%\end{flushright}
%\end{large}

%\begin{large}
%\begin{minipage}[t]{3cm}
%\textit{Laureando} \\
%\myName
%\end{minipage}
%\hfill
%\begin{minipage}[t]{5cm}
%\textit{Relatore} \\
%Prof. \myProf
%\end{minipage}
%\end{large}

\begin{large}
\begin{center}
\begin{tabularx}{0.9\textwidth}{l X r}
\textit{Laureando} & & \textit{Relatore}\\
\myName & & Prof. \myProf
\end{tabularx}
\end{center}
\end{large}


\vspace{40pt}

\line(1, 0){338} \\ 

\vspace{10pt}
\begin{large}
\textsc{\myAA}
\end{large}


\end{center}
\end{titlepage}

\blankpage
\input{inizio-fine/colophon}
%% !TEX encoding = UTF-8
% !TEX TS-program = pdflatex
% !TEX root = ../Lahmer_Abdelilah_tesi.tex
% !TEX spellcheck = it-IT

%**************************************************************
% Dedica
%**************************************************************
\cleardoublepage
\phantomsection
\thispagestyle{empty}
\pdfbookmark{Dedica}{Dedica}

\vspace*{5cm}

%\begin{center}
%Lorem ipsum dolor sit amet, consectetuer adipiscing elit. \\ \medskip
%--- Oscar Wilde    
%\end{center}

\medskip

\begin{center}
\flushright Dedicato alla mia famiglia e agli amici più stretti.
\end{center}

%% !TEX encoding = UTF-8
% !TEX TS-program = pdflatex
% !TEX root = ../Lahmer_Abdelilah_tesi.tex
% !TEX spellcheck = it-IT

%**************************************************************
% Ringraziamenti
%**************************************************************
\cleardoublepage
\phantomsection
\pdfbookmark{Ringraziamenti}{ringraziamenti}

\leavevmode	\newline

\begin{flushright}{
	\slshape    
	``He who does not thank people, does not thank God''} \\ 
	\medskip
    --- Prophet Muhammad (PBUH)
\end{flushright}


\bigskip

\begingroup
\let\clearpage\relax
\let\cleardoublepage\relax
\let\cleardoublepage\relax

\chapter*{Ringraziamenti}

 \noindent \textit{Innanzitutto vorrei ringraziare i miei genitori, Najeh e Latifa, per avermi accompagnato e concesso di arrivare fin qui. Grazie inoltre alla mia intera famiglia per il sostegno e per essermi sempre stati vicini.}\\

\noindent \textit{Ringrazio i compagni di studi per tutti i bellissimi anni passati insieme, in particolare i colleghi di Answer Group.}\\

\noindent \textit{La più sentita gratitudine inoltre agli amici più stretti, in particolare ringrazio Hamza, Abdourahmane, Amir, Mustafa e Sara per tutto il loro affetto e sostegno ricevuto.}\\

%\noindent \textit{Ringrazio Sopra Steria Group S.p.A. e tutti i dipendenti della sede di Padova per avermi accolto e seguito durante il tirocinio.}\\

\noindent \textit{Ringrazio sentitamente, infine, il prof. \myProf, relatore della mia Tesi, per l'aiuto, i preziosi consigli e la pazienza che mi ha dedicato per lo svolgimento del lavoro.}\\

\bigskip

\noindent\textit{\myLocation, \myTime}
\hfill \myName

\endgroup

%\blankpage
%\blankpage


% !TEX encoding = UTF-8
% !TEX TS-program = pdflatex
% !TEX root = ../Lahmer_Abdelilah_tesi.tex
% !TEX spellcheck = it-IT

%**************************************************************
% Sommario
%**************************************************************
\cleardoublepage
\phantomsection
\pdfbookmark{Sommario}{Sommario}
\begingroup
\let\clearpage\relax
\let\cleardoublepage\relax
\let\cleardoublepage\relax

\chapter*{Sommario}

Il presente documento riassume il lavoro svolto durante il periodo di stage, della durata di circa 300 ore, presso l’azienda Sopra Steria Group S.p.A con sede a Padova.
%La mission di Sopra Steria Group consiste nell'accompagnare e aiutare i suoi clienti a conseguire il successo attraverso il processo di trasformazione dei loro processi di business e dei loro sistemi informativi.
%Io sono stato inserito nella Divisione Servizi Finanziari, sezione che si occupa di sviluppo e manutenzione di sistemi bancari.
%Il tirocinio formativo e orientativo che mi è stato proposto ha avuto lo scopo di avviarmi verso la conoscenza della realtà lavorativa, approfondendo e verificando l'apprendimento ricevuto nel percorso degli studi con un'esperienza soggettiva legata direttamente alla realtà economica e produttiva del territorio, svolta nell'ambito di una realtà multinazionale.
%Nello specifico sono stato inserito in un gruppo di lavoro che opera su vari progetti di analisi e sviluppo, nonché manutenzione, di soluzioni bancarie per primari istituti di credito sul territorio italiano, quali Banca Popolare di Verona e Banco Popolare di Milano.
L’obiettivo minimo del tirocinio è stato l'acquisire padronanza dell'ambiente di sviluppo MAINFRAME, del linguaggio di programmazione COBOL e l'essere in grado di comprendere correttamente le analisi tecniche.
Obiettivo desiderabile è stato raggiungere anche una discreta autonomia nell'analisi di funzionalità e il concepimento, anche se parziale, di modalità di traduzione di queste in analisi tecnica.

%\vfill
%
%\selectlanguage{english}
%\pdfbookmark{Abstract}{Abstract}
%\chapter*{Abstract}
%
%\selectlanguage{italian}

\endgroup			

\vfill


% !TEX encoding = UTF-8
% !TEX TS-program = pdflatex
% !TEX root = ../Lahmer_Abdelilah_tesi.tex
% !TEX spellcheck = it-IT

%**************************************************************
% Sommario
%**************************************************************
%\cleardoublepage
\phantomsection
\pdfbookmark{Convenzioni tipografiche}{Convenzioni tipografiche}
\begingroup
\let\clearpage\relax
\let\cleardoublepage\relax
\let\cleardoublepage\relax

%\chapter*{Convenzioni tipografiche}
										%\chapter{Convenzioni tipografiche}
%\section*{Convenzioni tipografiche}

\leavevmode	\newline
\leavevmode	\newline
\leavevmode	\newline
\begin{Huge}
Convenzioni tipografiche
\end{Huge}
\leavevmode	\newline

Per la stesura del documento ho adottato le seguenti norme tipografiche:\\

\begin{itemize}
	\item L'utilizzo del \textit{corsivo} per le parole di ambito tecnico;
	\item L'utilizzo del \textit{corsivo} per i termini in lingua inglese che non dispongono di un corrispettivo termine in italiano, o che nel contesto in cui vengono utilizzati sia meglio adoperare il termine inglese; 
	\item L'indicazione con una G a pedice della prima occorrenza del capitolo di tutti i termini che necessitano di una spiegazione esplicita, definita nel glossario presente a fine documento;
	\item L'utilizzo del \textbf{grassetto} per evidenziare termini di rilievo nei paragrafi.
\end{itemize}


%\vfill
%
%\selectlanguage{english}
%\pdfbookmark{Abstract}{Abstract}
%\chapter*{Abstract}
%
%\selectlanguage{italian}

\endgroup			

%\vfill


\input{inizio-fine/indici}
\cleardoublepage

%**************************************************************
% Materiale principale
%**************************************************************
\mainmatter
% !TEX encoding = UTF-8
% !TEX TS-program = pdflatex
% !TEX root = ../Lahmer_Abdelilah_tesi.tex
% !TEX spellcheck = it-IT

%**************************************************************

\chapter{L'azienda}

%**************************************************************
\section{Profilo aziendale}
L'azienda presso la quale ho svolto il mio stage è Sopra Steria Group S.p.A, una delle imprese che propone una delle offerte più complete di servizi \textit{end to end} presenti oggi sul mercato.\\

\begin{figure}[H]
	\centering
   	\includegraphics[width=0.6\textwidth]{immagini/logo_azienda}
   	\caption{Logo di Sopra Steria Group S.p.A. - Fonte: \url{https://goo.gl/vbAJ6D}}
\end{figure}

Sopra Steria è infatti tra i leader europei in ambito di trasformazione digitale, propone una delle più complete offerte di servizi di \textit{Consulting}, \textit{Systems Integration}, \textit{Software Development} e \textit{Business Process Services} presenti oggi sul mercato.
Essa spazia su diversi mercati come \textit{Fashion}, \textit{Insurance}, \textit{Banking}, \textit{Retail}, \textit{Energy}, Aeronautica, Industria e Servizi, Sanità, Settore pubblico, Difesa e Trasporti.\\

Sopra Steria Group è partner di riferimento delle principali aziende ed organizzazioni pubbliche e private proponendo progetti di trasformazione di successo per affrontare al meglio le sfide di business più critiche e complesse, combinando un'alta qualità dei servizi erogati, valore aggiunto e innovazione.\\

L'azienda conta più di 40.000 collaboratori in più di 20 paesi, vanta inoltre un fatturato di 3,7 miliardi di euro nel 2016. Nello specifico opera sul territorio italiano con più di 800 risorse distribuite nelle sue sedi di Ariano Irpino (AV), Assago (MI), Asti, Collecchio (PR), Padova e Roma, fatturando circa 56,9 milioni nel 2016.\\

\begin{figure}[htbp]
\centering
\begin{minipage}[c]{.40\textwidth}
\centering\setlength{\captionmargin}{0pt}%
\includegraphics[width=0.6\textwidth]{immagini/dipendenti_paesi_fatturato_mondo}
\caption{Dati generali Sopra Steria nel Mondo}
\end{minipage}%
\hspace{10mm}%
\begin{minipage}[c]{.40\textwidth}
\centering\setlength{\captionmargin}{0pt}%
\includegraphics[width=1.2\textwidth]{immagini/mappa_italia_fatturato}
\caption{Dati generali Sopra Steria in Italia}
\end{minipage}
\caption{Informazioni generali Sopra Steria Group S.p.A. - Fonte: documento interno aziendale}
\end{figure}


Il gruppo è il risultato di una fusione avvenuta nel 2014 ad opera di due aziende, Sopra Group SA e Groupe Steria SCA, comunemente chiamate Sopra e Steria, fondate rispettivamente nel 1968 e 1969. Ad oggi l'azienda si presenta internamente ben strutturata in \textit{Business Unit}\footnote{Da questo punto per indicare il termine \textit{Business Unit} userò a volte anche il corrispettivo termine in italiano, ovvero Divisione.} relative agli ambiti di sviluppo e adotta una politica di \textit{recruiting} che mira alla competenza dei dipendenti da cui deriva la qualità dei prodotti, punto di forza dell'azienda.\\

Io sono stato inserito nella divisione "793 - Servizi Finanziari e Assicurazioni" della sede di Padova, nata negli ultimi anni a partire da pochi dipendenti e che ora conta circa 20 dipendenti solo per la sede in cui ho avuto il piacere di collaborare, senza contare i colleghi situati nelle sedi di Collecchio e Roma che cooperano anch'essi agli stessi progetti per la stessa divisione.\\
In particolare il mio ruolo è stato quello del sviluppatore \textit{host} e Analista Funzionale. \\
	Sono stato affiancato quindi da vari colleghi a seconda della tecnologia o conoscenza che dovevo apprendere. Più nello specifico sono stato affiancato dal mio collega Stefano Gori e dal mio tutor aziendale Marco Valentino, uno dei principali sviluppatori \textit{host} e manager di prossimità di questa sede, per l'apprendimento dei linguaggi COBOL\glossario\ e JCL\glossario. Sono stato invece affiancato dalle mie colleghe Chiara Maccotta e Francesca Constantini per l'apprendimento dei concetti teorici in ambito economico.\\ % essenziali per poter aver un quadro generale di quale sia lo scopo pratico del prodotto su cui si lavora e di una base teorica che permetta di raggionare sui risultati ottenuti dopo implementazioni e rispettivi risultati ottenuti.
	
	In quella che è la struttura aziendale i due ruoli per cui sono stato formato, ovvero quello di sviluppatore \textit{host} e Analista Funzionale, sono due mansioni a stretto contatto, in certi casi ricopribili ibridamente anche dallo stesso soggetto. \\ Più nel dettaglio l'incarico di Analista Funzionale consiste nel raccogliere i requisiti del cliente, tramite incontri ed interviste, ed il relazionarsi con esso al fine di capire ciò di cui ha bisogno. Con la redazione dei vari documenti di analisi questa figura deve essere in grado di portare all'interno del gruppo di sviluppo tutte le informazioni necessarie e sufficienti al successo della soluzione che verrà adottata. Nel contesto aziendale e di progetto in cui sono stato inserito io, avendo a che fare con istituti di credito, chi ricopre questo ruolo generalmente ha un \textit{background} più economico che tecnico, o in alternativa ha una pluriennale esperienza come sviluppatore tale per cui riesce comunque a tradurre le nozioni finanziarie in nozioni tecniche indirizzate agli sviluppatori. \\ Per quanto riguarda la mansione di sviluppatore \textit{host} invece consiste nel tradurre ciò che elaborano gli analisti in soluzione software mediante applicazioni in linguaggio COBOL. Generalmente chi ricopre questo ruolo ha un \textit{background} più tecnico che economico.\footnote{Da questo punto le descrizioni delle caratteristiche aziendali faranno riferimento alla \textit{Business Unit} in cui sono stato inserito, ovvero la "793 - Servizi Finanziari e Assicurazioni". Questo dato che ognuna di esse opera secondo logiche, ambiti e tecnologie differenti.}


%**************************************************************
\section{Prodotti e servizi offerti}
	
	\subsection{Prodotti}
	
	Nei mercati francesi, dove l'azienda è radicata, è attiva la vendita di prodotti bancari già pronti e configurabili in poco tempo presso i clienti. In Italia la situazione è differente e per i principali clienti, nell'ambito finanziario, raramente si vendono pacchetti di prodotti finiti ma si adotta una politica di personalizzazione secondo le esigenze del cliente. I prodotti principali offerti dalla \textit{Business Unit} in cui sono stato formato sono quindi riassumibili in:
		
	\begin{itemize}
		\item Applicazioni \textit{web} per la gestione di finanziamenti bancari, assieme alla relativa evoluzione e manutenzione;
		\item Programmi \textit{host} di gestione dati, gestione dei calcoli assieme alla consistenza e persistenza dei dati che vengono elaborati;
		\item Applicazioni \textit{web} utilizzabili dalla clientela degli istituti di credito.
	\end{itemize}
	
	\subsection{Servizi}
	
	Il sistema di gestione per la qualità dei servizi offerti ai clienti di Sopra Steria Group è certificato ISO 9001:2015 ed è annualmente sottoposto a verifiche da parte di un ente accreditato di terza parte.\\

	\begin{figure}[H]
	\centering
   	\includegraphics[width=0.85\textwidth]{immagini/ISO9001}
   	\caption{Politica della qualità di Sopra Steria Group S.p.A. - Fonte: documento interno aziendale}
	\end{figure}
	
	
	I principali servizi erogati dalla divisione per i clienti sono:
	
	\begin{itemize}
		\item Consulenza in ambito informatico per l'ampliamento ed il soddisfacimento della clientela da parte dei commerciali;
		\item Analisi delle necessità del cliente e dei conseguenti requisiti software;
		\item Progettazione e realizzazione di nuove applicazioni o nuove funzionalità di applicativi già in uso;
		\item Verifica e collaudo del software prodotto con finale rilascio nei sistemi del cliente;
		%\item Manutenzione dei contenuti proposti al cliente in un'ottica a lungo termine;
		\item Formazione del personale utilizzatore del prodotto, in particolare in seguito al rilascio di nuove funzionalità;
		\item Assistenza degli istituti bancari in caso si verifichi qualsiasi tipo di problema inerente al prodotto fornito.
	\end{itemize}	 

%**************************************************************
%\newpage
\section{Processi aziendali}

	\subsection{Organizzazione interna}
	
	L'organizzazione interna di Sopra Steria Group S.p.A. è un'organizzazione prettamente gerarchica	 che si sviluppa non solo a livello di direzione e sedi italiane ma a livello mondiale. Infatti le grandi dimensioni dell'azienda implicano questa forte strutturazione interna, assieme ad un'attenta gestione delle attività di coordinamento. Durante lo stage ho avuto modo di osservare molti degli aspetti di questo complesso sistema, anche inizialmente con la richiesta di tirocinio, che ho visto risalire lungo la gerarchia piramidale sino al consenso del direttore di \textit{Business Unit} e poi ritornare a livelli più bassi, dove una delle assistenti delle risorse umane mi ha notificato l'accettazione della richiesta.\\
	
	Una delle prime cose che si imparano di quest'azienda è la sua propensione alla cura dei rapporti con il cliente, poiché vengono offerte numerose sessioni	di consulenza. La filosofia è quella di collaborare per aiutarli a trasformare i loro sistemi informativi e, grazie all'esperienza del settore, offrire valore aggiunto mediante le soluzioni. Per raggiungere tale scopo il gruppo ha stretto delle \textit{partnership} strategiche con Microsoft, IBM, Oracle e HP. La missione principale del gruppo è di industrializzare e ottimizzare le proprie operazioni per migliorare la competitività e le  \textit{performance} in un'ottica a lungo termine.\\
	
	Un altro aspetto a cui l'azienda tiene in particolar modo è la gestione delle risorse umane. Sostenere lo sviluppo dell'evoluzione di queste ultime è considerata una priorità per il successo aziendale e per mantenere un alto livello di soddisfazione e di motivazione dei dipendenti. Per questo Sopra Steria si impegna a conoscere i profili e le competenze di ciascun collaboratore, al fine di poter offrire agli stessi prospettive di crescita e percorsi di carriera in grado di soddisfare sia le loro aspettative che il mercato. A tale scopo l'azienda organizza annualmente il cosiddetto PAP, acronimo che sta per \textit{Plan Annuel de Progression} ovvero Piano di Progressione Annuale, che serve appunto a favorire la crescita delle risorse; durante questo evento infatti i dipendenti vengono esaminati seguendo uno schema ben preciso e se le conoscenze dimostrate sono tali da potersi assumere responsabilità più grandi, questa possibilità di crescita viene valutata. Il colloquio PAP permette inoltre al collaboratore e al suo manager di prossimità di fare un bilancio del periodo appena trascorso e di fissare gli obiettivi e il piano di crescita per l'anno a venire. Durante questo evento si ha anche la possibilità di evidenziare l'esigenza di formazione e di accompagnamento per una gestione coerente della carriera. Il piano di crescita così elaborato rappresenta un impegno reciproco assunto dal collaboratore e dal suo superiore.% Fase importante della vita professionale, il colloquio PAP è un momento privilegiato di dialogo con l'azienda attraverso delle figure di riferimento, quali i Manager di Prossimità.\\
	
	%La crescente complessità dei progetti, la molteplicità degli interlocutori e le esigenze di alto livello dei clienti sono tali da comportare rischi considerevoli per l'azienda. L'intervento della direzione legale si rende pertanto necessario per difendere al meglio gli interessi del gruppo, tutelando i rapporti contrattuali con i clienti e le fasi di contenzioso, i rapporti con le software house, le parti terze, i partner e i fornitori e le acquisizioni o cessioni di attività.
	
	Per quanto riguarda il governo e la gestione del gruppo, i diversi livelli di poteri decisionali, sia a livello funzionale che produttivo, sono distribuiti nella gerarchia operativa oltre che nella direzione. Alla base di ciò, un'organizzazione complessa si ramifica nelle varie nazioni in cui l'azienda si estende, delegando l'amministrazione di questi filoni e di altri reparti di supporto a manager selezionati. A livello più basso si collocano le \textit{Business Unit}, ovvero le varie divisioni aziendali adibite all'erogazione di determinate tipologie di prodotti e servizi identificate anche in base al mercato di riferimento. Anche queste ultime risultano distribuite nel territorio, in ogni filiale infatti possono coesistere più reparti.\\
	
	\begin{figure}[H]
	\centering
	   	\includegraphics[width=0.8\textwidth]{immagini/Mercati_Principali}
	   	\caption{Suddivisione aree di mercato dell'azienda - Fonte dati: documento interno aziendale}
	\end{figure}
	
	Nella divisione in cui sono stato collocato vi sono diverse figure che si occupano dei vari processi di produzione. In ordine gerarchico è presente un direttore di \textit{Business Unit} per l'amministrazione delle risorse della divisione, i \textit{Project Manager} per la gestione dei progetti e dei loro costi, gli Analisti Commerciali che si occupano delle relazioni con i clienti, i team di Analisti e Consulenti che si occupano dei requisiti del cliente e i team di sviluppo software, suddivisi in sviluppatori \textit{web} e sviluppatori \textit{host}.\\

	%\subsection{Modello Incrementale}
	\subsection{Ciclo di sviluppo}
	
	Il ciclo di sviluppo software adottato da Sopra Steria nella divisione "793 - Servizi Finanziari e Assicurazioni" dove sono stato inserito è un'implementazione del modello incrementale, questa scelta è dovuta al fatto che l'azienda tratta per la maggior parte dei casi progetti di grandi dimensioni, il più delle volte progetti già avviati, che richiedono aggiunte sulla base delle funzionalità essenziali già sviluppate. Questo modello si caratterizza inoltre per la capacità di adattamento a molteplici tipologie di problemi.\\
	
	I punti di forza del procedimento incrementale sono i seguenti:
	\begin{itemize}
		\item L'integrazione delle parti del sistema è distribuita nel tempo e non collassata nelle fasi finali;
		\item La suddivisione in sottoinsiemi della realizzazione del problema comporta una migliore conoscenza di esso e una sua gestione più semplificata;
		\item Ogni incremento porta valore aggiunto, con lo sviluppo di nuove funzionalità e il soddisfacimento di alcuni requisiti;
		\item Ad ogni incremento si guadagnano esperienza e affidabilità, riducendo i rischi di fallimento;
		\item Le funzionalità essenziali sono sviluppate nei primi incrementi e attraversano più fasi di verifica, diventano quindi più stabili con ciascuna iterazione; questo sistema di \textit{rilasci multipli e successivi} permette anche al proponente di seguire in maniera attiva la prosecuzione del progetto avendo un'idea concreta del prodotto in corso di sviluppo.
	\end{itemize}
	
	 Questo modello si caratterizza inoltre per la capacità di adattamento a molteplici tipologie di problemi, in aggiunta si presta bene alle necessità dell'azienda perché i clienti richiedono che vengano effettuati lavori di manutenzione e amplificazione definibili in attività distinte, assimilabili facilmente tramite un ciclo di sviluppo ad incrementi. In figura vengono rappresentate le fasi del modello.\\
	
	\begin{figure}[H]
		\centering
	   	%\includegraphics[width=1\textwidth]{immagini/modello_incrementale}
	   	\includegraphics[width=1\textwidth]{immagini/ModelloIncrementale}
	   	%\caption{Modello di sviluppo incrementale}
	   	\caption{Modello di sviluppo incrementale - Fonte: \url{https://goo.gl/ZcNU8P}}
	\end{figure}
	%\newpage
	
	Il modello incrementale è un ciclo di sviluppo definito dallo standard ISO 12207 che combina la logica del modello a cascata, dove ogni fase è rigidamente sequenziale, e la filosofia iterativa della prototipazione.\\
	
	È prevista una prima fase di analisi dei requisiti fondamentali e di progettazione architetturale intesa a stabilire le fondamenta del software. Tale fase è essenziale per definire i successivi incrementi e non si ripete.\\
	
	Le fasi successive di realizzazione incrementale vera e propria, possono ripetersi più volte e mirano ad attività di progettazione di dettaglio, codifica e test, in cui vengono trattati prima i requisiti obbligatori, poi quelli facoltativi passando per quelli desiderabili. Le implementazioni subiscono i trattamenti di integrazione e collaudo, successivamente avviene un eventuale rilascio.\\

	È prevista la prototipazione delle nuove funzionalità che si vanno ad implementare per la validazione complessiva del sistema, reiterando 
	alle fasi di progettazione e realizzazione in caso di errori o problematiche. In questo modo è possibile di volta in volta acquisire maggiore competenza riguardo al problema, riducendo i rischi successivi e le tempistiche globali di produzione software.
	
	%\subsection{}
	\paragraph{Il modello incrementale in Sopra Steria}
\leavevmode	\newline \newline
	Ogni ciclo di incremento inizia con la raccolta e l'analisi dei requisiti presso il cliente, che espone le sue necessità tramite riunioni oppure mediante opportuna documentazione.\\
	
	Questo procedimento di raccolta dei requisiti e stesura dei documenti di analisi avviene generalmente dalle figure di Analisti Funzionali. Questi si occupano di raggruppare i requisiti in macro attività, calcolare le tempistiche necessarie per il loro completamento e stilare i documenti di \textit{Analisi Funzionale}. Dopo la stesura di tale documento, questo viene esposto al proponente al fine di approvazione. In caso il cliente ritenga che questo documento non sia adatto alle richieste o che ci siano delle mancanze viene notificato e gli analisti provvedono alle dovute correzioni. Questo procedimento di correzione della documentazione e attesa di approvazione può ripetersi sino alla giunta dell'accettazione da parte del proponente.\\
	
	Una volta approvato il documento entra in gioco la figura del \textit{Responsabile Commerciale} che presenta ai clienti un preventivo per l'implementazione delle funzionalità richieste.\\ %Pure questo preventivo passa all'approvazione del proponente e solo
	
	Una volta ottenuta l'accettazione dell'offerta commerciale si da il via alla stesura del documento di \textit{Analisi Tecnica} in cui si evince la progettazione di dettaglio. Tale documentazione risulta necessaria ai vari team di sviluppo per la comprensione e l'applicazione delle implementazioni richieste, ma non indispensabile per alcuni di essi.\\
		
	Gli analisti rimangono a disposizione degli sviluppatori anche nelle fasi successive per eventuali chiarimenti e specificazioni, in modo da non rallentare o interrompere le fasi successive. I documenti vengono inviati ai team competenti a cui sono state attribuite le macro attività e da quel momento inizia la realizzazione. Tali gruppi di lavoro possono risultare distribuiti nelle varie sedi del territorio italiano, perciò sono previste molte comunicazioni telefoniche o tramite posta elettronica e occasionali trasferte, al fine di allineare le procedure di sviluppo o rendere noto quando è possibile procedere con determinate modifiche. \\
	
	L'evoluzione degli incrementi software attraversa ambienti distinti. Esistono in particolare i seguenti ambienti:
	\begin{itemize}
		\item \textbf{Sviluppo}: ambiente di programmazione locale, qui avviene l'implementazione delle modifiche software;
		\item \textbf{Integrazione}: in questo ambiente vengono raccolte le implementazioni delle attività e si verifica che non generino conflitti, mediante test di non regressione\footnote{Con test di non regressione si intendono i test tali per cui si prova che, in eseguito alla modifica di una parte P del sistema S, la modifica di P non abbia introdotto errori né in P né alle altre parti di S che hanno relazione con P.}, garantendo la stabilità del sistema;
		\item \textbf{Collaudo}: ambiente di validazione delle funzionalità complessive del software, utilizzato anche per dimostrare al cliente la loro consistenza;
		\item \textbf{Produzione}: questo ambiente varia per ogni cliente o applicazione sviluppata e rappresenta lo stato finale del prodotto in cui viene effettivamente utilizzato dal cliente.
	\end{itemize}	
	
	È responsabilità del programmatore che prende in carico lo sviluppo delle funzionalità dichiarare il loro completamento, almeno a livello di prototipo, per rilasciarlo in integrazione. Determinati team si occupano poi di testare l'applicazione nelle sue nuove funzioni, accertando il soddisfacimento dei requisiti ed eventualmente contattando gli analisti per eventuali modifiche progettuali. In caso di problematiche le modifiche vengono respinte in ambito di sviluppo altrimenti vengono approvate per il collaudo. In collaudo è possibile utilizzare le funzioni sviluppate da altri team e validare il lavoro svolto per presentarlo poi al cliente, rilasciando in produzione la nuova versione del software.
	
	%\subsection{Strumenti a supporto di processi e servizi}
	\subsection{Tecnologie e strumenti a supporto di processi e servizi}

	Nel corso dello stage sono stati utilizzati numerosi strumenti di supporto per facilitare lo svolgimento delle diverse attività, dai tool di gestione delle basi di dati all'analisi del codice scritto passando per la gestione di progetto. Al fine di fornire ai suoi collaboratori tutti gli strumenti utili a rendere al meglio, Sopra Steria Group fornisce un portale da cui chiunque può scaricare o richiedere uno strumento, debitamente giustificato; tutto ciò affinché persista un alto livello di soddisfacimento e di motivazione dei dipendenti, che come abbiamo detto parlando di organizzazione interna è un punto sul quale l'azienda conta molto. Il portale porta il nome di \textbf{IT CORP} che abbinato al portale principale aziendale denominato Face2Face e il servizio sottostante di IT Request permettono di ottenere in qualsiasi momento applicativi o qualsivoglia strumento necessario nell'ambiente lavorativo.

	\begin{figure}[H]
		\centering
	   	\includegraphics[width=1\textwidth]{immagini/ITCorp}
	   	\caption{La pagina iniziale di IT CORP - Fonte: Portale interno dell'azienda}
	\end{figure}
	
	La scelta degli strumenti è basata sulla pluriennale esperienza del team, che ha scelto gli strumenti con cura, dopo un attento studio delle funzionalità offerte di ognuno, attenendosi a vari fattori tra cui il supporto presente online e facilità di utilizzo, in modo da assicurare un'alta qualità dei processi. Di seguito saranno descritte le principali funzionalità e caratteristiche dei vari strumenti.\\

	Parlando di ambiente lavorativo della divisione "793 - Servizi Finanziari e Assicurazioni" e strumenti a supporto di processi e servizi all'interno di essa non si può non distinguere inizialmente e prima di tutto le due macrocategorie di ruoli che un dipendente di questa business unit può assumere, ovvero la categoria degli sviluppatori \textit{Web} e quella degli sviluppatori \textit{Host}. Ognuna di queste due famiglie opera secondo politiche diverse, utilizzando processi e strumenti ben distinti. Di seguito quindi saranno descritte le principali funzionalità e caratteristiche dei vari strumenti specificando se si tratta di uno strumento a supporto della prima categoria di sviluppatori o della seconda.
	
	\subsubsection{Linguaggi}
	
	I linguaggi di programmazione utilizzati dagli sviluppatori facenti parte della divisione dove sono stato inserito sono molteplici, essendo che quest'ultima lavora su vari progetti in distinte sedi italiane. Discuterò quindi solo delle tecnologie utilizzate all'interno del progetto su cui lavora il team di cui ho fatto parte durante lo stage.
		
	\subsubsection{Linguaggi lato Host}
	Durante il periodo di stage gran parte delle attività di formazione si sono concentrate sui linguaggi da usare lato Host, ovvero i linguaggi COBOL e JCL.\\
	
	Il linguaggio \textbf{COBOL} è un linguaggio che risale agli ultimi anni '50, il suo nome è un acronimo che sta per COmmon Business-Oriented Language. Come si capisce dal nome esteso questo linguaggio è prettamente orientato al business, la traduzione del nome esteso infatti è "\textit{linguaggio comune orientato alle applicazioni commerciali}", e questo è infatti l'uso che se ne fa generalmente, ovvero programmi di gestione di sistemi bancari e assicurativi.\\

	I principali vantaggi di cui godeva il COBOL rispetto ai linguaggi che gli facevano concorrenza al tempo sono stati: 
	\begin{itemize}
		\item L'aritmetica con il punto decimale fisso, fattore molto utile nei programmi con funzioni di contabilità, che troviamo molto nel dominio bancario e in quello delle assicurazioni;
		\item Una maggiore velocità di input/output;
		\item Una sintassi con un'ottima leggibilità, conferita dal fatto che è simile a quella della lingua inglese;
		\item La capacità di gestione di enormi volumi di elaborazione con facilità.
	\end{itemize}		

	Con lo sviluppo e il perfezionarsi di questo linguaggio versione per versione, si è giunti a quella del 2002 con la quale il COBOL subiva una svolta significativa, ovvero il supporto della programmazione orientata agli oggetti.\\
	 
	Il linguaggio \textbf{JCL}, invece, è un linguaggio che anch'esso risale alla seconda metà del XX secolo ed è un acronimo che sta per Job Control Language. Il JCL è un linguaggio di \textit{scripting} generalmente utilizzato nei sistemi operativi IBM per eseguire (in gergo lanciare) una procedura batch\glossario\ su un sistema in genere mainframe. Nello specifico, lo scopo del JCL è quello di dire quali programmi eseguire, usando quali file di input e quali generare in output. L'uso che se ne fa nel contesto aziendale in cui sono stato inserito è principalmente quello di programmare e regolare l'esecuzione di programmi che generalmente vengono eseguiti periodicamente.

	\subsubsection{Linguaggi lato Web}

	Dovendo sviluppare anche l'applicativo web gli sviluppatori web su questo
fronte hanno scelto di adottare la piattaforma Java per il web (\textbf{Java EE}) e ovviamente le tecnologie standard relative alla presentazione e al comportamento delle pagine, ovvero \textbf{HTML}, \textbf{CSS} e \textbf{JavaScript}.\\

\begin{figure}[htbp]
\centering
\begin{minipage}[c]{.40\textwidth}
\centering\setlength{\captionmargin}{0pt}%
\captionsetup{width=1.2\linewidth}
\includegraphics[width=0.7\textwidth]{immagini/JavaEE}
\caption{Logo Java EE - Fonte: \url{https://goo.gl/wckR1M}}
\end{minipage}%
\hspace{15mm}%
\begin{minipage}[c]{.40\textwidth}
\centering\setlength{\captionmargin}{0pt}%
\captionsetup{width=1.25\linewidth}
\includegraphics[width=1\textwidth]{immagini/HTML5_CSS_JavaScript}
\caption{Logo HTML5, CSS3 e \\JavaScript - Fonte:\\ \url{https://goo.gl/A72tDP}}
\end{minipage}
\caption{Tecnologie utilizzate dagli sviluppatori Web}
\end{figure}
		
	L'edizione di Java che utilizzano gli sviluppatori Web è Java Platform Enterprise Edition, comunemente chiamata Java EE, che è un'estensione di Java SE (Standard Edition) e rappresenta una piattaforma di sviluppo software molto usata per applicazioni d'impresa.\\
	
	La versione di HTML utilizzata invece è la HTML5, che è l'ultima \textit{release} di HTML, risponde alle esigenze moderne ed alle aspettative dei siti web. Una delle caratteristiche principali di questa ultima edizione è il concetto di markup semantico, ovvero la capacità di fornire informazioni sul contenuto che descrive un dato tag. Oltre a questa particolarità, un'altra qualità che spicca è la capacità di adattarsi perfettamente, ovvero ad avere il medesimo comportamento sia su desktop che su mobile. \\
	
	Per tutti questi aspetti HTML5 è diventando un nuovo standard per gli sviluppatori web, tant'è che è diventato \textit{W3C Recommendation} dall'ottobre 2014.\\

	Per quanto riguarda il comportamento delle pagine web invece si è optato per l'uso della versione CSS3 per la gestione della formattazione delle pagine e di JavaScript per le validazioni e controlli \textit{client-side}.
		
	\subsubsection{Database}
	Il salvataggio dei dati per le applicazioni in ambito bancario e assicurativo avviene generalmente tramite DBMS\glossario\ relazionali come DB2 di IBM, Microsoft SQL Server e MySQL di Oracle. Il team di sviluppo in base anche ai calcolatori a disposizione della banca, anch'essi della IBM, ha scelto di utilizzare il \textbf{DB2}, che è nato nel 1983 ma tutt'oggi è uno tra gli RDBMS\glossario\ più usati, specie in questo settore. In origine era nato come DBMS per i mainframe CICS\glossario\, poi si è diffuso su diversi tipi di server. Per questo banche e assicurazioni, enti che esistono da molto prima della nascita del DB2, inizialmente hanno adottato questa tecnologia mediante sistemi EIS\glossario\ implementati in linguaggio COBOL che tutt'oggi gli forniscono le funzionalità necessarie senza il bisogno di adottare tecnologie più moderne e sviluppate secondo le esigenze dei più recenti paradigmi di programmazione.\\
	
	L'amministrazione delle basi di dati avviene tramite uno strumento denominato \textbf{DBeaver}, che è un applicazione gratuita multipiattaforma per sviluppatori, programmatori, amministratori di dabases e analisti.%Supporta più tipologie di database come MySQL, PostgreSQL, Oracle, DB2, SQL Server, MS Access e molti altri.% Teradata, SQLite, Sybase, Firebird, MariaDB, Derby 

	\begin{figure}[H]
	\centering
	\includegraphics[width=0.85\textwidth]{immagini/DBeaver_ss}
	\caption{Schermata dello strumento di amministrazione dei databases DBeaver - Fonte: \url{https://github.com/serge-rider/dbeaver}}
	\end{figure}

	\subsubsection{Ambienti di sviluppo ed emulatori}
	\label{Ambienti di sviluppo ed emulatori}

	Anche parlando di ambienti di sviluppo è importante distinguere quelli usati lato sviluppatori \textit{web} e quelli usati lato sviluppatori \textit{host}.\\
		
	Il principale ambiente di sviluppo adottato per le applicazioni web è \textbf{Eclipse}. Esso racchiude la globalità delle caratteristiche necessarie ad uno sviluppatore in questo ambito. Rappresenta un'ottima soluzione e agevolazione per il processo di sviluppo, in quanto offre funzionalità di collegamento ai sistemi di versionamento, \textit{debugging} del codice \textit{runtime}, oltre alle molteplici caratteristiche offerte dai comuni editor di testo orientati allo sviluppo dei sorgenti software.\\
	
%	\begin{figure}[H]
%		\centering
%	   	\includegraphics[width=0.7\textwidth]{immagini/ambienti_sviluppo}
%	   	\caption{I vantaggi dell'uso di Eclipse in collaborazione con RTC}
%	\end{figure}
	
	Altri programmi di supporto sono invece i diversi browser in cui bisogna testare il funzionamento delle pagine web tra cui Internet Explorer, Firefox e Chrome e gli editor di testo utili in situazioni dov'è richiesta più praticità come Notepad++.\\

	Il principale ambiente di sviluppo adottato lato host invece è \textbf{ISPF}\glossario. Esso include vari tool e funzionalità per la gestione dell'intero processo di sviluppo dei programmi host; dalla creazione dei programmi al versionamento degli stessi, dall'analisi statica del software in fase di compilazione alla gestione della base di dati tramite lo strumento QMF\glossario.

	\begin{figure}[H]
		\centering
	   	\includegraphics[width=0.80\textwidth]{immagini/ISPF}
	   	\caption{Schermata iniziale ambiente di sviluppo ISPF}
	\end{figure}

	Durante il periodo di stage ho avuto modo di imparare ad usare gran parte dei \textit{tool} che mette a disposizione ISPF, assistendo anche all'utilizzo di alcune funzionalità a cui sono abilitate solo un determinato tipo di utenze, ovvero le cosidette \textit{utenze di produzione}, che sono quelle con il totale accesso anche ai database e ai sistemi collocati dal cliente, e non solo a quelli di test su cui lavorano il resto degli sviluppatori.\\
	
	La struttura del menu di questo ambiente di sviluppo è indicativamente come illustrato nel seguente schema:
	
	\begin{figure}[H]
		\centering
		\captionsetup{width=0.70\linewidth}
	   	\includegraphics[width=0.80\textwidth]{immagini/ISPF_menu_structure}
	   	\caption{Struttura menu ambiente di sviluppo ISPF - Fonte: \url{https://goo.gl/GX6P9V}}
	\end{figure}

	Per utilizzare ISPF in azienda ho utilizzato l'emulatore \textbf{Quick3270 Secure} che è un potente ed affidabile emulatore di terminali IBM 3270 e IBM 5250. Utilizzabile su sistemi operativi Windows questo programma permette infatti di connettere il proprio computer ai sistemi IBM zSeries (S/390) e iSeries (AS/400).

	\subsubsection{Gestione di progetto}

	A supporto della gestione delle attività progettuali Sopra Steria mette a disposizione dei suoi dipendenti un portale comune che permette la gestione dei gruppi di lavoro. Oltre alle funzionalità di gestione di progetto questo portale permette anche l'organizzazione della comunità aziendale e favorisce il dialogo organizzato dipendente-azienda, seppur virtuale.\\
	
	Il portale aziendale, \textbf{Face2Face}, gestisce molteplici attività e problematiche. Tramite esso i dipendenti sono tenuti a riportare settimanalmente le proprie attività di lavoro e gli ambiti di progetto al fine di inviare i dati alla direzione, permettendole di coordinare le risorse a disposizione. Lo strumento consente inoltre di consultare le news aziendali e gli eventi organizzati.

	\begin{figure}[H]
		\centering
	   	\includegraphics[width=1\textwidth]{immagini/Face2Face}
	   	\caption{La Home Page di Face2Face - Fonte: portale interno dell'azienda}
	\end{figure}
		
	Face2Face è accessibile anche dall'esterno della rete aziendale tramite un portale online predisposto dall'azienda. In questo modo viene facilitato il lavoro in trasferta dei dipendenti.\\
	
	\subsubsection{Documentazione}
	
	Nell'arco della mia permanenza in azienda per lo stage ho avuto modo di vedere quella che è la documentazione che la prassi aziendale vuole che venga redatta. Per ogni attività risultante dall'analisi dei requisiti, infatti, vengono stilati due documenti: l'\textit{Analisi Funzionale} e l'\textit{Analisi Tecnica}. In fase di rilascio delle funzionalità richieste durante la fase di analisi vengono redatti invece i documenti di \textit{Collaudo} e quello di \textit{Rilascio}.\\
	
	Il primo documento, ovvero quello di Analisi Funzionale, affronta i requisiti ad alto livello, enunciando le principali funzionalità ed i cambiamenti rispetto alla versione attualmente in produzione dell'applicativo.\\
	
	\leavevmode	\newline

	Le sezioni principali di questo documento sono:
\begin{itemize}
	\item Matrice dei requisiti: enunciato discorsivo per introdurre il problema proposto;
	\item Descrizione funzionale: per spiegare il comportamento dell'applicativo lato web, presentando possibilmente anche un'anteprima delle pagine che verranno aggiunte;
	\item Casi oggetto di collaudo: qui vengono indicate le componenti che saranno analizzate in fase di  \textit{testing} per verificarne il corretto comportamento;
	\item Dettaglio tecnico: qui vengono enunciate le componenti software che saranno modificate o aggiunte, senza entrare nel dettaglio di come tali modifiche andranno apportate.
\end{itemize}

	Il secondo documento invece, ovvero quello di Analisi Tecnica, affronta nel dettaglio gli aspetti tecnici che vanno modificati o aggiunti trattando principalmente i programmi COBOL lato host, dai quali poi anche i programmatori web possono individuare i parametri da utilizzare nelle richieste via rete per recuperare i dati e quindi allinearsi.\\
	
	Il documento di Analisi Tecnica, diversamente da quello di Analisi Funzionale non sempre viene steso visto che questo non ha obbligo di approvazione da parte del cliente e quindi, in particolari casi, gli sviluppatori con più esperienza sono in grado di progettare e sviluppare la parte tecnica in autonomia senza l'aiuto di quest'ultimo.\\
	
	Al termine dello sviluppo dei requisiti, ad avvenuta validazione, vengono inoltre redatti i documenti di Collaudo e di Rilascio, da consegnare al cliente per accertare i lavori eseguiti e il rilascio delle nuove funzionalità.\\
			
	Il software utilizzato per la produzione dei documenti è \textbf{Microsoft Word}, i cui formati sono standard sia per l'azienda che per i clienti.
	
	
	\subsubsection{Sistemi di versionamento}

	Come precedentemente accennato nella sezione riguardante gli ambienti di sviluppo [\ref{Ambienti di sviluppo ed emulatori}], lo strumento \textbf{ISPF} permette, in un certo qual modo, il versionamento dei programmi sorgenti lato host. Questo infatti tiene traccia di ogni modifica apportata ai programmi e consente quindi di ripristinare una versione di un dato modulo utilizzando un progressivo numerico, identificativo di una certa versione del software, che si incrementa ogni qualvolta questo viene modificato e compilato.\\
	
	Per quanto riguarda il lato web, invece, nel progetto su cui lavora il team in cui sono stato inserito il sistema di versionamento del software utilizzato è \textbf{RTC} (Rational Team Concert, IBM), che è costruito su IBM Jazz, una piattaforma estensibile che aiuta i team ad integrare i task attraverso il ciclo di vita del software. RTC dispone di un'architettura client-server e permette ai team di sviluppo di tenere traccia del loro lavoro in modo intuitivo.
	
	\subsubsection{Sistemi operativi}
	
	Per le postazioni di sviluppo è previsto un sistema centralizzato di utenze a cui è attribuito un proprio ambiente di lavoro ed uno spazio assegnato a cui accedere da qualsiasi computer aziendale.\\
	
	Nelle macchine aziendali è consueto l'utilizzo di Microsoft Windows 7 come sistema operativo primario per ovviare a discrepanze nelle postazioni dei diversi dipendenti e per omogenizzare ulteriormente il processo di sviluppo. Questo comporta una buona soluzione per concentrarsi unicamente sul proprio lavoro e avere al contempo una garanzia nell'utilizzo quotidiano.
	
%**************************************************************

%\section{Clientela e trasformazione digitale}
\section{Clientela ed innovazione}

	\subsection{Clientela target}
	
	La clientela target della \textit{Business Unit} "793 - Servizi Finanziari e Assicurazioni" di Sopra Steria sono principalmente gruppi bancari e assicurativi. Essi solitamente necessitano una qualche forma di innovazione o evoluzione che gli garantisca continuità di produzione ma anche i corretti adeguamenti previsti dai cambiamenti legislativi.\\
	
	 Per permettere questo, gli analisti si incaricano di entrare in contatto con i responsabili ICT\glossario\ della società cliente, dai quali poi si ricavano le diverse richieste implementative;  che possono andare dalla variazione di qualche caratteristica alla creazione di funzionalità completamente nuove.\\

	%\subsection{Principali clienti e progetti}
	
	Visto il successo conquistato con il passare degli anni Sopra Steria Group si è sempre fatta strada tra i brand più importanti in vari settori, tra i quali i seguenti, classificati per aree di mercato.
	\begin{figure}[H]
	\centering
   	\includegraphics[width=0.7\textwidth]{immagini/principali_referenze}
   	\caption{Principali clienti di Sopra Steria Group S.p.A. - \\Fonte: documento interno aziendale}
	\end{figure}

	I progetti di maggior impatto, invece, per cui Sopra Steria è riuscita ad ottenere il privilegio di approvvigionamento e conseguente implementazione sono stati molteplici, tra questi citiamo i seguenti.
	\begin{figure}[H]
	\centering
   	\includegraphics[width=0.76\textwidth]{immagini/Progetti_Importanti}
   	\caption{Principali progetti di Sopra Steria Group S.p.A. - \\Fonte: documento interno aziendale}
	\end{figure}

	
	\subsection{Innovazione}
	
	L'innovazione e le soluzioni software richieste dalle banche e assicurazione non sono mai stati termini accostabili. %, soprattutto negli ultimi anni.
	Se da un lato le tecnologie stanno subendo una rivoluzione importante dall'altro lato gli istituti di credito preferiscono avere sistemi funzionanti e garantiti anche se il mantenimento degli stessi richiede somme non da poco. Se da un lato si parla di migrazione sul  \textit{Cloud} dei sistemi informativi delle aziende all'avanguardia dal lato degli istituti di credito si parla al più di cambio di versione dei  \textit{Framework} utilizzati lato  \textit{front-end}\glossario.\\

	\begin{figure}[H]
	\centering
	\includegraphics[width=0.85\textwidth]{immagini/VignettaCobol}
	\caption{Vignetta sull'uso del COBOL - Fonte: \url{https://goo.gl/dVnwEg}}
	\end{figure}
	
	L'innovazione in ambito bancario e assicurativo rappresenta infatti un'ostacolo non indifferente, questo tipo di enti sono da sempre legati a tecnologie primordiali come il linguaggio COBOL e la relativa implementazione in mainframe CICS.\\
	
	Per Sopra Steria, che fa della trasformazione digitale e innovazione un suo punto di forza, questo rappresenta una sfida, l'azienda infatti desidera mettersi in gioco offrendo le soluzioni adeguate, tenendo conto però delle priorità del cliente e delle sue possibilità. Queste caratteristiche sono molto ricercate dalle aziende che vogliono rinnovarsi, trasformando i loro processi e servizi nel mondo digitale, adeguandosi ai moderni canoni di utilizzo e facendosi avanti nei mercati, con la possibilità di offrire prodotti di maggiore qualità e raggiungere molti più clienti.\\
	
	Per quanto riguarda il progetto su cui lavora il team in cui sono stato inserito l'unico fattore di innovazione riguarda il lato  \textit{front-end} dell'applicazione, l'evoluzione infatti da questo lato si fa vedere mediante l'utilizzo di tecnologie moderne adatte alla presentazione dei contenuti	nel web, tecnologie che sono maggiormente soggette a spinte evolutive dovute alla modernizzazione degli standard.             % Azienda
% !TEX encoding = UTF-8
% !TEX TS-program = pdflatex
% !TEX root = ../Lahmer_Abdelilah_tesi.tex
% !TEX spellcheck = it-IT

%**************************************************************

\chapter{La scelta del progetto}

\section{Interesse aziendale negli stage}
%In questa sezione parlerò del tipo di relazione che lega l'azienda al mondo degli stage e il loro interesse per questi. Parlerò inoltre della presenza di altri ex-stagisti che hanno fatto un mio percorso simile e sono stati assunti.

Viste le dimensioni e l'attuale espansione dell'azienda, dovuta anche al momento di buona ripresa economica che sta vivendo l'Italia, Sopra Steria Group S.p.A. è sempre più alla ricerca di nuove figure da inserire nelle proprie divisioni che operano in diversi ambiti e settori.\\

Le modalità con cui la società attua alla ricerca di risorse sono molteplici, queste infatti possono andare dal reclutamento attraverso agenzie per il lavoro al reclutamento tramite \textit{social network} orientati al lavoro (come ad esempio LinkedIn\footnote{LinkedIn. URL: \url{https://goo.gl/nc4mVh}}), passando per il reclutamento per mezzo di eventi organizzati dalle università o dagli enti pubblici per agevolare l'incontro tra le aziende e gli studenti.\\

Quest'ultimo tipo di evento rappresenta la modalità con cui sono entrato in contatto con Sopra Steria, ovvero grazie al progetto Stage-IT\footnote{Stage-IT. URL: \url{https://goo.gl/UvdwLK}}, nella sua XIV edizione che si è tenuta in data 5 Aprile 2017, dove le aziende hanno avuto modo di esporre i propri progetti e gli studenti le proprie ambizioni a far parte di essi. STAGE–IT infatti è un'occasione di conoscenza reciproca per permettere agli studenti di avvicinarsi al mondo del lavoro e alle imprese di presentare la realtà in cui operano, illustrando le tematiche proposte per stage, con specifico riferimento al settore “\textit{Information and Communication Technology}" (ICT\glossario).\\

	\begin{figure}[H]
		\centering
	   	\includegraphics[width=0.5\textwidth]{immagini/StageIT}
	   	\caption{StageIT 2017 - Fonte: \url{https://goo.gl/TqV3A6}}
	\end{figure}


La politica aziendale prevede una durata di sei mesi per completare il ciclo di stage, due di questi considerati curricolari in accordo con l'Università degli Studi di Padova e quattro extracurricolari. Successivamente, nella maggior parte dei casi, vi è la propensione all'assunzione, in quanto la nuova risorsa si considera pronta per essere effettivamente inserita nei team di sviluppo o di analisi.\\

Il tirocinio è quindi visto dall'azienda come uno strumento utile a contribuire alla selezione di nuovi talenti e verificare che da entrambe le parti vi sia un interesse a proseguire il rapporto lavorativo.\\

L'interesse dell'azienda per gli stagisti e per l'incanalamento di questi nel mondo del lavoro mi è stato confermato una volta approdato nella sede per i colloqui conoscitivi, qui infatti ho avuto modo di rincontrare un collega dell'università che anch'esso aveva seguito lo stesso percorso prima di me, circa un'anno prima. Questo collega infatti dopo aver concluso lo stage curricolare, conseguito la laurea ed aver concluso anche il periodo di stage extracurricolare è stato assunto a tempo indeterminato; l'unica differenza dal percorso che stavo per intraprendere io però è stato soltanto il fatto che questo collega dopo la fine dello stage bimestrale ha chiesto il cambio di \textit{Business Unit} passando alla "791/792/796 - Industria e Servizi" dove ha trovato un inquadramento come sviluppatore di applicazioni \textit{web} e \textit{mobile}. Vista questa esperienza l'impressione che l'azienda aveva trasmesso inizialmente si è sempre più confermata e personalmente guardavo con fiducia l'avvenire all'interno di Sopra Steria.

\section{Il progetto all'interno dell'azienda}
%In questa sezione parlerò del progetto che mi è stato proposto dell'azienda.

Gli stage che le aziende proponevano durante l'evento Stage-IT erano prevalentemente orientati a progetti contenuti che secondo le aziende erano fattibili nel \textit{range} di tempo che il corso di studi ci impone, ovvero un minimo di 300 ed un massimo di 320 ore. Dopo aver dato una buona impressione all'evento sono rimasto in contatto con l'amministrazione delle risorse umane di Sopra Steria Group S.p.A, che alla fine del mese di Aprile mi ha convocato per un colloquio collettivo assieme ad altri otto candidati. Anche in questa sede l'impressione che sono riuscito a dare e quella che sono riuscito a percepire da parte dell'azienda è stata più che positiva. Il 17 Maggio infatti iniziavo il mio percorso presso Sopra Steria.\\

Durante il colloquio conoscitivo collettivo le domande sono state varie, in particolare una domanda ha fatto la differenza, ovvero: "Quali sono le vostre ambizioni per il futuro?". Inizialmente lo stage ideale che vedevo in azienda consisteva in un progetto nel ramo della programmazione \textit{mobile}, proiettandomi già in un ottica di stesura di una relazione di fine stage in cui sono portato a presentare un vero e proprio progetto possibilmente, che sarebbe stato sicuramente fattibile in quel ramo. La mia risposta però ha fatto sì che tutto questo venisse sconvolto. La mia replica infatti è stata grossomodo: "La mia ambizione per il futuro è quella di diventare un \textit{Project Manager} oppure un Analista Funzionale, certamente dopo aver fatto un po' di anni di gavetta per acquisire esperienza". Il colloquio collettivo infine si è concluso con la mia convocazione ad un incontro assieme al \textit{Project Manager} di prossimità della sede che mi ha illustrato in maniera generica il progetto su cui lavora il team, ovvero ELISE\glossario, e in cosa consisteva la figura di \textit{Project Manager} e Analista Funzionale in quell'ambito lavorativo. A primo impatto il ramo del \textit{banking} sembrava interessante ma rimanevo comunque aggrappato al mio stage ideale su quello del \textit{mobile}; cosa che però non ha considerato l'amministrazione delle risorse umane, che mi ha indirizzato alla \textit{Business Unit} "793 - Servizi Finanziari e Assicurazioni" con il fine di formare una figura di Analista Funzionale.\\

Lo stage, quindi, così come stava per essere intrapreso, consisteva nella formazione per una mansione più che consistere in un progetto.\\

Assieme al tutor che mi era stato assegnato si sono poi decisi gli obiettivi e a grandi linee il percorso che dovevo seguire. Quest'ultimo prevedeva la mia formazione anche sulle tecnologie e modalità di sviluppo utilizzate lato \textit{back-end}\glossario\ al fine di avere un'ottica più ad ampio raggio sull'intero funzionamento del sistema, sia lato teorico che tecnico.
%Dopo una prima fase di formazione sulle tecnologie ed una di esercitazione secondo i metodi aziendali di sviluppo, quindi, il lavoro di stage si è concentrato su questo software, chiamato ELISE. Ho portato avanti il suo ampliamento aggiungendo le nuove funzionalità il cui sviluppo mi è stato assegnato.
	
%	\begin{figure}[H]
%		\centering
%	   	\includegraphics[width=1\textwidth]{immagini/fasi_progetto}
%	   	\caption{Le fasi principali del progetto di stage}
%	\end{figure}


\subsection{ELISE: Extended Loans Integrated System}
In questa sezione parlerò abbastanza superficialmente del prodotto su cui la divisione in cui sono stato inserito lavora, ovvero ELISE, che è un applicazione web con scopo la gestione di finaziamenti.
		
\section{Il mio stage}
In questo inizio di sezione parlerò un po' di StageIT e dei progetti proposti dalle aziende in occasione di tale evento.

\subsection{Proposte di stage}
In questa sottosezione parlerò delle varie proposte di stage che ho ricevuto e dei progetti che proponevano.

\subsection{Motivo della scelta}
In questa sottosezione parlerò della scelta di stage che ho fatto e del motivo di tale scelta.

\subsection{Obiettivi del progetto}
In questa sottosezione parlerò degli obiettivi del progetto di stage che ho scelto.

\subsection{Vincoli del progetto}
In questa sottosezione parlerò dei vincoli imposti sia dall'azienda che dal cliente per il progetto di stage che ho scelto.

\subsection{Obiettivi personali}
In questa sottosezione parlerò degli obiettivi che mi sono imposto iniziando il progetto di stage.
             % Progetto
%% !TEX encoding = UTF-8
% !TEX TS-program = pdflatex
% !TEX root = ../tesi.tex
% !TEX spellcheck = it-IT

%**************************************************************
\chapter{Stage}
\label{cap:stage}
%**************************************************************

%\intro{Breve introduzione al capitolo}\\

%**************************************************************
\section{Pianificazione del lavoro}

Per raggiungere gli obiettivi pianificati nel piano di stage e rispettare i requisiti minimi imposti dall'Università, io e il tutor aziendale abbiamo previsto 320 ore di lavoro, distribuite in 8 settimane da 40 ore ciascuna. Ho iniziato lo stage il 26/09/2016 e ho terminato, a causa di un giorno di festività, nel lunedì 21/11/2016, rimanendo in linea con quanto preventivato inizialmente, senza incorrere in imprevisti.

	\subsection{Definizione del piano di lavoro}
	
	Un mese prima dell'inizio dello stage ho redatto un Piano di Lavoro, definendo gli obiettivi e la pianificazione delle attività a granularità settimanale, elencando nel dettaglio i compiti da svolgere per ogni fase. Ho specificato inoltre le modalità di interazione col tutor, revisioni di avanzamento e una previsione delle competenze guadagnate dallo stagista alla fine delle attività. Le fasi identificate in tale documento sono:
	
	\begin{itemize}
		\item \textbf{Formazione teorica}: durante la prima fase del percorso formativo il tirocinante viene introdotto alle modalità di approccio alla programmazione web mediante: 
			\begin{itemize}
				\item utilizzo dell'ambiente di sviluppo Eclipse per implementazioni e \textit{debug};
				\item studio della piattaforma Java EE, integrazione di \textit{application server} e utilizzo di diverse JVM;
				\item implementazione di interfacce grafiche tramite JSP;
				\item studio di framework di uso comune come Struts, Hibernate e Maven;
				\item sviluppo delle funzionalità mediante l'utilizzo di linguaggi Java e JavaScript.
			\end{itemize}
		\item \textbf{Formazione pratica}: in questa fase lo stagista inizia a lavorare a stretto contatto con il resto del gruppo di lavoro, imparando come affrontare correttamente le attività da un punto di vista sia tecnico sia analitico. Vengono utilizzati i sistemi di versionamento per lo sviluppo di programmi di esempio, dai più semplici (solo Java) ad una completa applicazione web in tecnologia Java EE, secondo la metodologia di lavoro corretta. Questo periodo termina con l'analisi della struttura di una complessa applicazione reale.
		
		\item \textbf{Sviluppo su applicazione reale}: analisi e implementazione di modifiche relative ad attività reali dell'applicazione ELISE, progetto di stage, in affiancamento al gruppo di lavoro. Il tirocinante si impegna ad analizzare e successivamente stimare le attività che gli verranno assegnate portandole a compimento al pari di una qualsiasi delle figure del team nelle giuste tempistiche.		
	\end{itemize}
	
	\begin{figure}[H]
		\centering
	   	\includegraphics[width=1\textwidth]{immagini/tabella_gantt}
	   	\caption{Pianificazione delle attività di stage}
	\end{figure}
		
	Questa pianificazione dettagliata mi ha permesso di distribuire il carico di lavoro e di verificare l'allineamento tra il lavoro effettivamente svolto e il lavoro pianificato, al termine di ogni settimana.	
	
	\subsection{Livello di autonomia}
	
	Durante lo svolgimento dello stage il tutor aziendale è stato disponibile per ogni mia necessità, fornendomi consigli riguardo le tecnologie che andavo ad affrontare, specie nelle	prime fasi. Ho lavorato in un ambiente che mi consentiva di essere a stretto contatto con lui in modo da favorire l'interazione e garantire il raggiungimento degli obiettivi prefissati.\\
	
	Sono state effettuate verifiche di avanzamento sia settimanali che giornaliere, quando ritenuto necessario, con relativa revisione dei prodotti per assicurarsi del corretto punto di completamento rispetto alla pianificazione concordata in partenza.\\
	
	Una volta presa confidenza con gli ambienti e la strumentazione aziendale, ho avuto modo di agire liberamente nelle mie attività, ricevendo dal tutor solo indicazioni sulla strada da percorrere e sui vincoli da rispettare. Successivamente sono stato capace di lavorare in maniera autonoma, com'era desiderabile aspettarsi, ricevendo solo qualche saltuaria delucidazione.	
	
%**************************************************************
\section{Studio degli strumenti di sviluppo}

Con il mio arrivo in azienda è iniziata la prima fase del lavoro di stage, ovvero lo studio delle tecnologie adottate dalla divisione Servizi Finanziari per l'adempimento dei suoi scopi. L'obiettivo era quello di conferirmi una formazione teorica da consolidare in seguito con delle prove pratiche a scopo esercitativo.\\

Nelle prime settimane ho quindi studiato i linguaggi, le tecniche di progettazione e gli ambienti di sviluppo che assieme al tutor aziendale avevo pianificato di trattare nel piano di lavoro. Nelle settimane successive avrei poi messo in pratica e ottenuto padronanza di tali tecnologie secondo le metodologie aziendali, approfondendo le conoscenze su tali argomenti, in preparazione alla fase di sviluppo sull'applicazione ELISE.

	\newpage

	\subsection{Tecnologie utilizzate}

	\subsubsection{Piattaforma web}
	Java Platform Enterprise Edition, o Java EE, è un'estensione di Java SE (Standard Edition) e rappresenta una piattaforma di sviluppo software molto usata per applicazioni d'impresa.\\
	
	Java EE è sviluppato mediante il Java Community Process, ovvero un processo che permette ad una comunità di esperti industriali, organizzazioni (commerciali e \textit{open source}) e un'infinità di individui di dare il loro contributo seguendo determinati standard. Esso fornisce gli strumenti utili per la programmazione web, tra cui un ambiente \textit{runtime} e librerie utili allo sviluppo di applicazioni distribuite, scalabili, affidabili e sicure che prevedono Java come linguaggio di programmazione primario.\\
	
	\begin{figure}[H]
		\centering
	   	\includegraphics[width=1\textwidth]{immagini/architettura_javaEE}
	   	\caption{Architettura di un'applicazione sviluppata in Java EE}
	\end{figure}
	
	Utilizzando Java EE risulta fondamentale l'utilizzo di un \textit{application server} (o \textit{servlet}\glossario\ \textit{container}) per l'esecuzione e la distribuzione dell'applicativo nei diversi nodi della rete.
	Si tratta di un ambiente che estende le funzionalità offerte da un normale Web Server, ovvero il paradigma client-server e la comunicazione dei contenuti mediante protocolli web. Esso è strutturato nei diversi livelli architetturali (architettura multi-tier) di cui un'applicazione web ha bisogno per eseguire:
	
	\begin{itemize}
		\item \textbf{Presentation layer}: rappresenta la logica di presentazione delle pagine web dell'applicazione;
		\item \textbf{Business layer}: strato della logica funzionale dell'applicazione, utile per la generazione di contenuti dinamici;
		\item \textbf{Persistent layer}: ovvero la gestione dei dati e della loro persistenza.
	\end{itemize}
	
	Sono disponibili diversi \textit{application server} tra cui JBoss (RedHat) e WebSphere (IBM) e servlet container come Tomcat (Apache).\\
	
	La piattaforma Java EE adotta una convenzione per la configurazione mediante XML delle componenti dell'applicazione, in particolare 
	il file web.xml (contenuto in \textit{WebContent/WEB-INF/}) contiene le impostazioni principali riguardanti la gestione delle richieste inviate al server.\\
	
	\begin{figure}[H]
		\centering
	   	\includegraphics[width=1\textwidth]{immagini/web_xml}
	   	\caption{Un esempio di configurazione imposta tramite il file web.xml - Fonte: ambiente di sviluppo in Eclipse}
	\end{figure}
	
	Quando si compila un applicazione Java EE viene generato un JAR\glossario\ (Java Archive) di tipo EAR (Enterprise Archive) o WAR (Web Archive), questi file vengono eseguiti rispettivamente dall'\textit{application server} e dal \textit{servlet}\glossario\ \textit{container}, due framework che si occupano quindi di avviare l'applicazione. 
	
	\subsubsection{Presentazione}
	
	Per la creazione delle interfacce grafiche delle applicazioni durante lo stage, ho utilizzato gran parte delle tecnologie standard, ovvero HTML, CSS per l'aspetto e la struttura e JavaScript per il comportamento. Tali tecnologie erano già di mia conoscenza e non richiedevano troppo impegno da parte mia.\\
	
	I contenuti generati con queste tecnologie, però, sono stati inglobati in un ambito di cui io non ero ancora a conoscenza: le pagine JSP.\\
	
	JSP (Java Server Pages) è una tecnologia di programmazione web in Java EE per lo sviluppo della logica di presentazione delle applicazioni, eseguito tipicamente fornendo contenuti dinamici in formato HTML e inglobando le tecnologie citate prima. Al momento della compilazione del software, tali pagine vengono lette dal compilatore JSP e trasformate nelle apposite classi Java Servlet\glossario , ovvero una specifica estensione di Java pensata per l'utilizzo web.
	
	\begin{figure}[H]
		\centering
	   	\includegraphics[width=1\textwidth]{immagini/jsp_compile}
	   	\caption{Flusso di compilazione di una JSP}
	\end{figure}
	
	\begin{figure}[H]
		\centering
	   	\includegraphics[width=1\textwidth]{immagini/jsp_comunications}
	   	\caption{Rappresentazione della cominicazione client-server in ambito JSP}
	\end{figure}	
	
	All'interno di una JSP viene definito l'HTML in cui viene immerso il codice Java sottoforma di \textit{scriptlet}. Il tutto viene poi compilato e come risultato viene prodotta una Java Servlet\glossario . In particolare questo tipo di classi fornisce al programmatore la possibilità di manipolare la comunicazione client-server mediante appositi oggetti. A questo punto, il codice HMTL viene stampato in pagina mediante l'oggetto \textit{HttpServletResponse}.\\
	
	Queste pagine si basano anche su un insieme di librerie di tag JSTL (Java Standard Tag Library), con cui possono essere invocate funzioni predefinite sotto forma di classi Java. In aggiunta, permette di creare librerie di nuovi tag che estendono l'insieme dei tag standard (JSP Custom Tag Library).
	
	\subsubsection{Framework}
	Durante il lavoro di stage ho sempre utilizzato il framework Struts, in particolare per lo sviluppo delle nuove funzionalità per l'applicazione ELISE dell'ultima fase.\\
	
	Si tratta di uno strumento utile alla creazione di applicazioni sviluppate secondo Java EE. Nella fase di studio delle tecnologie ho avuto modo di comprendere la sua forza, anche grazie alle mie conoscenze, acquisite durante il corso di studi. Struts infatti estende le Java Servlet\glossario , implementando il design pattern MVC\glossario\ (Model-View-Controller), definendo una solida struttura per il software che lo adotta. 
	
	\begin{figure}[H]
		\centering
	   	\includegraphics[width=1\textwidth]{immagini/MVC_struts}
	   	\caption{Struttura di un'applicazione sviluppata mediante il design pattern imposto da Struts}
	\end{figure}	
	
	L'utilizzo di questo framework permette lo sviluppo di applicazioni web di notevoli dimensioni, inoltre agevola la suddivisione dello sviluppo del progetto fra i vari dipendenti. I programmatori web e i vari gruppi di sviluppatori possono quindi gestire in parallelo e autonomamente la loro parte del progetto.\\
	
	Questo risulta maggiormente utile se si associano le proprie attività di sviluppo ad un sistema di versionamento, dove poter integrare le porzioni di software, ben strutturato secondo il framework.
	
	\subsubsection{Properties}	
	Data la grandezza del progetto ELISE ed il suo elevato numero di pagine, è stato adottato un meccanismo per la gestione delle \textit{label}, ovvero le scritte statiche da presentare in pagina.\\
	
	In particolare grazie alla tecnologia JNDI\glossario\ è possibile creare un vocabolario (memorizzato su file .properties) di stringhe Java, codificate mediante un nome identificativo, da richiamare nelle pagine JSP in modo da facilitare il mantenimento e la modifica di tale aspetto.\\
	
	\begin{figure}[H]
		\centering
	   	\includegraphics[width=1\textwidth]{immagini/JNDI}
	   	\caption{Sistema di vocabolario generato dall'uso di JNDI}
	\end{figure}	
	
	Ovviamente in questo modo più pagine possono riferirsi alla stessa stringa utilizzando lo stesso identificativo, nel momento in cui un label deve essere modificato basterà cambiarlo nel vocabolario e tutte le pagine che si riferiscono ad esso verranno aggiornate.
	
	\subsubsection{Logging}	
	Una tecnica molto utile che ho imparato durante dello stage è l'uso dei log, che nel percorso di studi avevo solo visto a grandi linee nel corso di Programmazione Concorrente e Distribuita.\\
	
	Tale tecnica prevede di utilizzare di un servizio, spesso una libreria esterna, per tracciare il flusso applicativo e stampare su file le informazioni rilevanti dei vari stati che il software ha incontrato durante la sua esecuzione. Nel mio caso è stato adottato Log4J, una libreria molto nota a questo scopo, utilizzata anche nel progetto ELISE.\\
	
	Essa permette di emettere delle notifiche nei vari punti del software, associandone il livello di importanza e livello generale dell'applicativo impostare il livello di debug da utilizzare durante l'esecuzione. Una volta lanciato, il programma produrrà il log contenente solo le notifiche che rispettano il livello selezionato, nascondendo quelle di minore importanza.\\
	
	\begin{figure}[H]
		\centering
	   	\includegraphics[width=1\textwidth]{immagini/log_levels}
	   	\caption{I diversi livelli di log selezionabili con la libreria Log4J - Fonte: tutorial online sull'uso della libreria}
	\end{figure}	
	
	In molte occasioni è stato necessario consultare i file di log per risalire ad eventuali errori oppure per comprendere meglio il comportamento dell'applicazione. Talvolta è stato scelto di modificare il livello di debug per analizzare più a fondo i messaggi prodotti dal software.
	
	\subsubsection{Stampe PDF}	
	Un'altra tecnica adotata dal progetto ELISE è la generazione dinamica di documenti PDF, da produrre nella filiale di competenza che li deve stampare per conto degli utenti allo	sportello.\\
	
	Per ottenere tutto questo sono previste delle classi Java specifiche dell'applicativo che si servono della libreria iText per l'inizializzazione di un documento ed il relativo riempimento con i giusti contenuti, sotto forma di tabelle, paragrafi, frasi, ecc. Anche questa libreria è tra le più usate per questo scopo, il che rende facile reperire guide e documentazione ufficiale su internet.
	
	\begin{figure}[H]
		\centering
	   	\includegraphics[width=1\textwidth]{immagini/itext}
	   	\caption{Illustrazione del processo di creazione dei documenti PDF mediante la libreria iText}
	\end{figure}
	
	\subsection{Alternative analizzate}
	
	\subsubsection{Hibernate}	
	Hibernate è un framework \textit{open source} per lo sviluppo di applicazioni Java. Fornisce un servizio di \textit{object-relational mapping} (ORM) ovvero gestisce la persistenza dei dati sul database attraverso la rappresentazione e il mantenimento su database relazionale di un sistema di oggetti Java. Come tale dunque, nell'ambito dello sviluppo di applicazioni web, si frappone tra il livello logico di business e quello di persistenza dei dati sul database.\\
	
	La funzione primaria di Hibernate quindi, è di mappare classi Java in tabelle del database e tipi di dati Java in tipi di dati SQL. Questo framework non è adottato dal team per questo progetto in quanto la persistenza dei dati mediante un sistema CICS\glossario\ non è supportata. Infatti il processo di codifica del	flusso dati per comunicare con l'ambiente \textit{host} è già ampiamente gestito mediante apposite classi Java che, con l'uso della Reflection e di mappe XML, si occupano di generare gli appositi JavaBean\glossario .
	
	\subsubsection{Spring}	
	Spring è un altro framework \textit{open source} per lo sviluppo di applicazioni su piattaforma Java. L'aspetto centrale nell'utilizzo di Spring è avere a disposizione il suo \textit{inversion of control container}, ovvero un ambiete che invita lo sviluppatore ad applicare il design pattern architetturale \textit{dependency injection}, permettendo di gestire in maniera consistente oggetti Java usando la Reflection.\\
	
	Il contenitore si occupa del ciclo di vita degli oggetti gestiti, detti beans, mediante una configurazione fornita su file XML. Oltre a questa funzionalità, il framework offre supporto ad attività di \textit{transaction management}, accesso ai dati, messaggistica e verifica. Spring non è adottato nello sviluppo di ELISE in quanto il progetto è già ampiamente strutturato secondo il framework Struts.
	
	\subsubsection{Maven}	
	Apache Maven è un framework per la gestione di un progetto software. Esso può gestire la compilazione del progetto, l'invio di segnalazioni e la documentazione basandosi sul concetto centrale di \textit{project object model} (POM), un file XML che descrive le dipendenze fra il progetto e le varie versioni di librerie necessarie.\\
	
	Maven effettua automaticamente il download di librerie Java e plug-in Maven dai vari repository definiti, scaricandoli in locale o in un repository centralizzato lato sviluppo. Questo permette di recuperare in modo uniforme i vari file JAR\glossario\ e di poter spostare il progetto indipendentemente da un ambiente all'altro avendo la sicurezza di utilizzare sempre le stesse versioni delle librerie.\\
	
	Questo framework è stato da me studiato solo a scopo formativo, in quanto l'ambito di progetto non richiedeva di usufruire delle funzionalità che esso offre. In particolare la portabilità del software trattato è garantita dall'utilizzo degli ambienti di collaudo del cliente, senza bisogno di renderlo compatibile con altre macchine.
	
	\subsubsection{SVN}	
	SVN (Subversion, Apache) è un sistema di versionamento, già descritto in sezione \ref{sistemi-versionamento}, alternativo ad RTC. È stato analizzato da parte mia a scopo formativo ed è utilizzato da parte dell'azienda per lo sviluppo altri progetti.

%**************************************************************
\section{Processo di sviluppo}

Al termine delle prime due fasi, come pianificato, il lavoro di stage aveva occupato un mese, metà del tempo a disposizione. Come ultima attività ho svolto un'analisi conoscitiva dell'applicazione ELISE, a cui avrei poi partecipato nello sviluppo. Mi preparavo quindi ad entrare nell'ultima fase, quella di implementazione delle espansioni su di un applicativo reale, per occupare il restante 50\% del lavoro.\\

	\subsection{Analisi dei requisiti}
	
	Come prima attività, per reperire il lavoro da svolgere, l'azienda esegue delle attività di consulenza e stabilisce i contratti con i clienti. Successivamente è tutto pronto, a livello burocratico, per concentrarsi sulle esigenze del cliente e raccogliere i suoi requisiti.\\
	
	Seguendo la struttura di governo aziendale, i team di analisti si confrontano con i responsabili tecnici e direttivi dell'ente che richiede il lavoro. In questa fase l'azienda si concentra molto per stabilire al meglio i requisiti del cliente, analizzando con cura i suoi bisogni e cercando di capire quali soluzioni possono soddisfarli. In questo modo garantisce che le successive attività di progettazione e codifica avvengano su una base solida, senza il rischio di dover ripetere attività successive a causa di errori a monte del progetto.\\
	
	 Questo processo avviene solitamente mediante trasferte da parte dei consulenti o, in caso di piccole attività, anche per via telefonica. Viene redatto, come output, un documento di programmazione delle macro attività da svolgere, che resta in mano ai project manager. Oltre a questo documento, gli analisti si occupano di redigere anche un'Analisi Funzionale per ognuna delle attività.\\
	 
	 Tale documento affronta i requisiti ad alto livello, enunciando le principali funzionalità ed i cambiamenti rispetto alla versione attualmente in produzione dell'applicativo. \\
	
	Le sezioni principali di questo documento sono:
	
	\begin{itemize}
		\item \textbf{Matrice dei requisiti}: enunciato discorsivo per introdurre il problema proposto;
		\item \textbf{Descrizione funzionale}: per spiegare il comportamento dell'applicativo lato web, presentando anche un'anteprima delle pagine che verranno aggiunte;
		\item \textbf{Casi oggetti di collaudo}: qui vengono indicate le componenti che saranno analizzate in fase di testing per verificarne il corretto comportamento;
		\item \textbf{Dettaglio tecnico}: qui vengono segnalate solo alcune particolarità e vengono enunciate le componenti software che saranno modificate o aggiunte.	
	\end{itemize}
	
	Questo documento di analisi risulta di primaria importanza per la programmazione delle interfacce. Sotto il punto di vista di un programmatore web infatti, è necessario comprendere come le pagine andranno strutturate e quale comportamento dovrà adottare l'applicazione in risposta alle interazioni con l'utente o l'ambiente \textit{host}.\\
	
	Nel mio lavoro di stage ho dovuto svolgere diverse attività e per ciascuna di esse avevo a disposizione tale documento, da poter consultare in ogni momento oltre ad un approfondito studio iniziale, prima di sviluppare alcuna componente.\\
	
	Per le attività a cui sono stato assegnato, mi sono occupato di analizzare il problema in questione e di catalogare i requisiti che mi venivano forniti dagli analisti, oltre a raccoglierne di nuovi. Ho attribuito ai requisiti la seguente forma:
	
	\begin{center}
		\texttt{Importanza Tipologia - Identificativo \hspace{1cm} Titolo}
	\end{center}
	
	Dove:
	
	\begin{itemize}
		\item \textbf{Importanza}: denota il peso del requisito di riferimento e ne stabilisce una priorità. Assume i valori:
		\begin{itemize}
			\item \textbf{O}: Obbligatorio, il requisito deve essere soddisfatto per considerare il progetto completo;
			\item \textbf{D}: Desiderabile, se il requisito viene implementato porta valore aggiunto al progetto ma non è strettamente necessario. 
		\end{itemize}
		
		\item \textbf{Tipologia}: indica la natura del requisito e assume i seguenti valori:
		\begin{itemize}
			\item \textbf{F}: Funzionale, rappresenta una funzionalità che il prodotto finale dovrà fornire, che sia essa espressa in forma generale a livello di sistema o espressa nel dettaglio delle componenti;
			\item \textbf{V}: Di vincolo, specifica un vincolo che il software deve rispettare;
			\item \textbf{P}: Prestazionale, si tratta di una caratteristica di performance che il prodotto deve soddisfare;
			\item \textbf{T}: Tecnico, ovvero un requisito implementativo che l'applicazione dovrà possedere per garantire determinate funzioni.
		\end{itemize}
		
		\item \textbf{Identificativo}: un numero, generato per incremento, che identifica il requisito in modo univoco se considerato assieme alla codifica della sua importanza e tipologia;
		
		\item \textbf{Titolo}: una breve descrizione del requisito e dei suoi riferimenti.
	\end{itemize}
	
%	Di seguito ho raccolto e catalogato in una tabella i requisiti relativi alle attività del mio lavoro di stage.\\
	
%	\begin{table}[H]
%		\def\arraystretch{1.2}
%		\begin{tabular}{ | p{2cm}  p{10cm} | }
%		
%		\rowcolor{Gray}
%		\hline \textbf{Codice} & \textbf{Titolo} \\ \hline
%		
%		%Stampe preventivo:
%			OF-01 & Fornire la stampa della nuova documentazione di preventivo \\ \hline
%			DF-02 & Generare un'apposita gerarchia di classi per la documentazione di preventivo \\ \hline		
%		%Scadenziario covenant:
%			OF-03 & Fonrire la possibilità di visualizzare le scadenze covenant \\ \hline
%			DF-04 & Fornire un calendario intuitivo a livello grafico per le scadenze covenant \\ \hline
%			OF-05 & Possibilità di visualizzare il dettaglio delle scadenze covenant per data \\ \hline		
%		%Composizione titolari:
%			OF-06 & Visualizzare i titolari di un finanziamento \\ \hline
%			OF-07 & Consentire il censimento delle categorie fiscali dei titolari \\ \hline
%			OF-08 & Collegare la pagina dei titolari alla procedura di erogazione \\ \hline
%			DF-09 & Possibilità di stampare una bozza PDF della documentazione titolari \\ \hline		
%		%Ricerca tassi:
%			OP-10 & Generare una maschera di filtro che agisca sulle date di validità dei tassi \\ \hline		
%		%Layout iframe:
%			OF-11 & Permettere lo spostamento della finestra per il web service esterno \\ \hline
%			OF-12 & Permettere il ridimensionamento della finestra per il web service esterno \\ \hline
%			OV-13 & Realizzare gli effetti per l'iframe usando il linguaggio JavaScript \\ \hline
%			DP-14 & Rendere fluidi e gradevoli i movimenti dell'iframe per il web service \\ \hline		
%		%Generale:
%			OV-15 & Seguire lo standard dettato dall'applicazione in termini di oggetti
%				grafici e creazione di funzionalità somiglianti a quelle presenti \\ \hline
%			OV-16 & Uso di itext per la generazione dei PDF di stampa \\ \hline
%			OV-17 & Seguire il template fornito dal cliente per i documenti di stampa \\ \hline	
%		
%		\end{tabular}
%		\vspace{1mm}
%		\caption{Tabella dei requisiti delle attività implementative di stage}
%	\end{table}

	In totale, per i miei ambiti ho raccolto 66 requisiti. Ho compreso le funzionalità richieste dal cliente e quelle che l'applicativo doveva implementare, analizzando le prime nel dettaglio per ricavare le seconde. Per fare ciò mi sono servito sia dei documenti di Analisi Funzionale, sia dell'esperienza del tutor e degli altri colleghi del team. Al termine delle fasi di analisi avevo i requisiti necessari per pilotare le successive attività di progettazione e sviluppo.\\
	
	Nel seguente grafico viene mostrata la suddivisione dei requisiti secondo la loro importanza.
	
	\begin{figure}[H]
		\centering
	   	\includegraphics[width=1\textwidth]{immagini/suddivisione_requisiti2}
	   	\caption{Suddivisione dei requisiti in base alla tipologia}
	\end{figure}
	
	Nel seguente grafico mostro invece la suddivisione dei requisiti secondo la loro tipologia.
	
	\begin{figure}[H]
		\centering
	   	\includegraphics[width=1\textwidth]{immagini/suddivisione_requisiti}
	   	\caption{Suddivisione dei requisiti in base alla tipologia}
	\end{figure}

	\subsection{Progettazione}
	
	Le scelte architetturali sono già state prese in passato alla nascita dell'applicativo e sono radicate in tutto il software. A differenza di queste, le scelte progettuali per le singole funzionalità, che si vanno ad aggiungere col passare del tempo durante la manutenzione e l'evoluzione del software, vengono stabilite ad ogni incremento del ciclo di sviluppo. \\
	
	Terminata la fase di analisi dei requisiti, i team di competenza dell'ambiente di gestione dei dati dell'applicazione si occupano di progettare i programmi lato \textit{host}. Questa attività influenza in modo decisivo anche il modo in cui verranno poi sviluppate le interfacce web per tali programmi.\\
	
	Come output di questa fase viene prodotto quindi il documento di Specifica Tecnica. Questo secondo documento affronta nel dettaglio gli aspetti tecnici trattando principalmente i programmi COBOL\glossario , dai quali poi anche i programmatori web possono individuare i parametri da utilizzare nelle richieste via rete per recuperare i dati. In questo modo lo sviluppo delle pagine web del software viene allineato con le implementazioni fatte su \textit{host}, permettendo una progettazione corretta anche dal punto di vista dei programmatori \textit{front-end}.\\
	
	\subsubsection{Progettazione architetturale}	
	
	Lo studio della struttura di ELISE mi ha permesso di conoscere i suoi funzionamenti, comprendendo quali scelte fossero adeguate o meno in tale sistema. L'applicazione, come menzionato, si basa sul framework Struts per la definizione delle proprie fondamenta. Il vantaggio più significativo portato da tale scelta è rappresentato dalla struttura che ne deriva. \\
	
	In particolare, Struts gestisce il flusso delle richieste in entrata all'applicazione e le identifica in \textit{action} da svolgere. Queste azioni vengono definite in un apposito file XML del framework, in cui si associa una classe Java ad un URL, in modo da collegare le interazioni con l'utente nelle pagine web con la logica di business dell'applicazione. Ogni azione ha poi una pagina JSP che rappresenta il risultato delle operazioni a basso livello, Struts si occupa di eseguirne il \textit{render} una volta ottenuti i dati di cui essa ha bisogno. Spesso tali dati sono incapsulati in JavaBeans\glossario .\\
	
	Tutto ciò non sarebbe possibile se la struttura imposta dal framework non seguisse una logica ben precisa. Il software infatti implementa in questo modo un \textit{design pattern} architetturale molto usato a livello strutturale: MVC\glossario . Per l'applicazione ELISE questo pattern è implementato secondo il modello web, ovvero a partire da una pagina JSP si richiama la logica di business (sottoforma di Enterprise JavaBeans) attraverso il server (rappresentato da classi Servlet). 
	
	\begin{figure}[H]
		\centering
	   	\includegraphics[width=1\textwidth]{immagini/diagramma_MVC}
	   	\caption{Diagramma delle classi per l'implementazione del \textit{design pattern} architetturale MVC - Fonte: slides del corso di Ingegneria del Software mod. B}
	\end{figure}
	
	In questo modo possono essere realizzate più viste grafiche che rimandano alla stessa componente Java, definendo le apposite parti che si occupano del controllo delle richieste e stabilendo i collegamenti nel file \textit{struts-config.xml}.
	
	\begin{figure}[H]
		\centering
	   	\includegraphics[width=1\textwidth]{immagini/struts_config}
	   	\caption{Esempio di configurazione delle componenti di Struts mediante file XML}
	\end{figure}	
	
	\subsubsection{Progettazione di dettaglio}
	
	Per le singole attività di manutenzione ed espansione di ELISE, la progettazione dell'applicativo avviene in base ai requisiti e agli scopi di tali modifiche. Le fasi di progettazione di dettaglio si susseguono infatti ad ogni incremento applicato sul software, allo scopo di definire in modo corretto le componenti utili al soddisfacimento dei requisiti. \\
	
	Io sono stato l'unico responsabile nel definire la struttura delle parti che componevano le mie attività. Il tutor mi ha insegnato precedentemente come agire e successivamente mi ha lasciato produrre in autonomia, revisionando il mio lavoro solo per eventuali accertamenti, ma è comunque rimasto sempre a mia disposizione.\\
	
	Per le mie attività ho quindi cercato di analizzare il problema e ideare delle soluzioni adatte all'applicazione di progetto. In particolare, come descritto, avevo già a disposizione una solida struttura e molte librerie utili. Si trattava di muovere i primi passi in quel sistema e progettare delle soluzioni che riutilizzassero quante più funzionalità già presenti. In tale ottica anche le mie soluzioni prodotte, se risultavano completamente nuove, andavano pensate per essere riutilizzate in futuro.\\
	
	Ho cercato di soddisfare i requisiti del cliente al meglio, senza fretta nel produrre subito una codifica delle soluzioni, ma pensando bene alla loro organizzazione. In questo senso mi sono impegnato a organizzare il codice in classi che rispettassero una solida gerarchia e fossero estendibili, per garantirne un'utilità futura. Ad esempio, per la gestione delle stampe dei preventivi e la documentazione della composizione dei titolari di un finanziamento, ho realizzato tale struttura per definire le varie tipologie di stampa. In tal modo ho raggruppato le parti in comune nelle classi poste in alto nella gerarchia.
	
	\begin{figure}[H]
		\centering
	   	\includegraphics[width=1\textwidth]{immagini/diagramma_stampe}
	   	\caption{Diagramma delle classi semplificato della gerarchia per le stampe}
	\end{figure}
	
	Un'altra tecnica che ho avuto modo di studiare e applicare in maniera approfondita è la Reflection\glossario . L'ho usata principalmente per rendere noto al sistema software quali parti delle mie componenti dover recuperare per svolgere determinate funzioni. Nel dettaglio ho modificato e creato dei nuovi file di configurazione XML, in cui ho definito i parametri e i nomi dei campi dei JavaBeans\glossario\ da utilizzare nella comunicazione con l'ambiente \textit{host} e nelle pagine dell'applicazione.\\
	
	Per quanto riguarda la tecnologia JSP, che racchiude diversi linguaggi in sé, è stato complicato definire uno stile generale per la creazione delle interfacce.\\
	
	A tale scopo ho cercato di separare i concetti e le utilità comprese in ogni pagina, progettando piuttosto delle classi Java esterne o diverse componenti JSP  da includere poi in un file principale. Ho creato quindi un'alta coesione delle componenti, identificando il più possibile il loro scopo e suddividendole in modo da non attribuire troppe responsabilità ad una singola parte. In questo modo andavo ad aumentare il livello di accoppiamento tra di esse, ma non risultava mai eccessivo da rappresentare uno svantaggio, in quanto il prodotto di grandi dimensioni diveniva così più comprensibile e strutturato, dando modo di riutilizzare le sue componenti.\\
	
	Come già reso noto, durante il corso dello stage ho avuto modo di applicare alcuni \textit{design pattern} studiati durante i miei studi all'Università. Un secondo pattern che desidero citare è il \textit{decorator pattern}, molto usato nella progettazione delle componenti dell'applicativo. Grazie a questo principio ho implementato delle soluzioni che aggiungevano delle funzionalità o dei contenuti a delle porzioni generiche del software.
	
	\begin{figure}[H]
		\centering
	   	\includegraphics[width=1\textwidth]{immagini/decorator_pattern}
	   	\caption{Diagramma delle classi del \textit{decorator pattern}}
	\end{figure}
	
	\subsection{Codifica}
	
	Una volta pianificate le azioni e progettate le componenti, per ogni attività, mi sono occupato di realizzare le soluzioni.	Dal punto di vista implementativo, avere alla base del proprio lavoro un framework che si occupa di strutturare le parti del sistema e categorizzarle secondo i loro scopi, è certamente un vantaggio.\\
	
	Le fasi di sviluppo sono state quelle che più mi hanno conferito padronanza degli strumenti e delle metodologie che dovevo imparare.\\
	
	Con l'aiuto del tutor facevo chiarezza sulle implementazioni da realizzare e sulle tecniche da impiegare, guadagnando poco alla volta autonomia e responsabilità delle mie azioni. \\	
	
	Ho creato diverse pagine JSP nel corso dello stage e ho notato che tale tecnologia può causare numerose ambiguità per uno sviluppatore esterno. Organizzando il codice in modo comprensibile ho definito una struttura e raggruppato le porzioni di codice di ogni tipologia di linguaggio.	Inizialmente definivo in una \textit{scriptlet} le porzioni di codice Java utile a recuperare le componenti di business, solitamente i JavaBeans\glossario . 
	
	\begin{figure}[H]
		\centering
	   	\includegraphics[width=1\textwidth]{immagini/codice_scriptlet}
	   	\caption{Esempio di codice Java incluso in una pagina JSP mediante \textit{scriptlet}}
	\end{figure}
	
	In seguito solitamente ho dato inizio alla struttura della pagina mediante i \textit{tag} personalizzati dell'applicazione, utilizzando quindi il suo \textit{template} per mantenere coerenza grafica e strutturale.\\
	
	Mi sono servito quindi dei tag standard di JSTL e ho sviluppato soluzioni con alcuni che invece sono realizzati da altri sviluppatori, come il \textit{display tag}, per la generazione di tabelle. In particolare questo tag implementa il \textit{decorator pattern}, ovvero sfrutta una classe Java appositamente realizzata per riempire la tabella con i relativi contenuti.
	
	\begin{figure}[H]
		\centering
	   	\includegraphics[width=1\textwidth]{immagini/display_tag}
	   	\caption{Esempio di codice JSP per la creazione di una tabella implementando il \textit{decorator pattern}}
	\end{figure}
	
	In seguito ho applicato le mie conoscenze di programmazione web, con i linguaggi HTML e CSS per completare la presentazione delle pagine.\\
	
	In tutte le interfacce che ho sviluppato, vi era il bisogno di definire anche un aspetto comportamentale dell'applicazione a livello client. In coda ai sorgenti delle pagine, infatti, ho inserito porzioni di codice JavaScript, utile a realizzare effetti grafici e richiamare determinate funzioni o procedure già presenti nel software.\\
	
	In tale contesto ho avuto modo di applicare la tecnica AJAX\glossario\ per recuperare determinate informazioni che risiedono nel server, senza imporre un ricaricamento della pagina. 
	
	\begin{figure}[H]
		\centering
	   	\includegraphics[width=1\textwidth]{immagini/codice_AJAX}
	   	\caption{Esempio di codice JavaScript per implementare la tecnica AJAX}
	\end{figure}

%**************************************************************
\section{Verifica e validazione}

Durante tutta la fase di implementazione delle funzionalità ho svolto anche attività di verifica e validazione. Durante la codifica ho compiuto tali attività mediante delle prove in locale, prima dei rilasci invece, attraverso un piccolo collaudo. Questo per verificare che tutti i requisiti fossero soddisfatti e che l'andamento dello sviluppo procedesse nella giusta direzione, al fine di portare in ambiente di integrazione delle soluzioni più corrette possibile ed evitare eventuali iterazioni nel ciclo di sviluppo.\\

	Io e il tutor abbiamo dato priorità a verifiche effettuate su casi di utilizzo reali dell'applicazione analizzando il funzionamento nei vari casi d'uso previsti per ogni attività, testandola quindi manualmente piuttosto che in automatico.\\
	
	Per il progetto ELISE infatti non era prevista la stesura a priori di test automatici da effettuare una volta prodotto il codice necessario, in quanto ogni attività di evoluzione, programmata per il software, si presenta ad un alto livello di granularità, garantendo una ristretta copertura di casi d'uso, facilitando quindi le attività di test.

	\subsection{Analisi statica}
	Lo strumento più usato per la verifica risulta essere l'ambiente di sviluppo Eclipse. Esso mette a disposizione dello sviluppatore un sistema di analisi statica che permette la visualizzazione di \textit{error} e \textit{warning} causati da problemi nel codice o utilizzi poco adatti del linguaggio.\\
	
	Per ogni codice sorgente che andavo ad estendere o creare, avevo la responsabilità di rimuovere tutti gli errori e limitare i \textit{warning} alle sole segnalazioni dovute all'utilizzo di versioni di JVM differenti. In tal modo si garantiva che il software non producesse malfunzionamenti dovuti ad uno scorretto uso della piattaforma Java EE.\\
	
	\begin{figure}[H]
		\centering
	   	\includegraphics[width=1\textwidth]{immagini/warning_eclipse}
	   	\caption{Interfaccia di Eclipse per la segnalazione di \textit{warning}}
	\end{figure}
	
	\vspace{5mm}
	
	\begin{figure}[H]
		\centering
	   	\includegraphics[width=0.8\textwidth]{immagini/error_eclipse}
	   	\caption{Interfaccia di Eclipse per la segnalazione degli \textit{error}}
	\end{figure}
	
		\vspace{5mm}	
	
	Nelle fasi di sviluppo, utilizzando Eclipse, ho inoltre avuto modo di usufruire delle sue funzionalità di \textit{debug} per il controllo di errori a tempo di esecuzione dell'applicazione. Questo mi ha permesso di scoprire errori comportamentali nel codice ma anche di comprendere i flussi intrapresi dal software durante il suo utilizzo.\\
	
	Oltre a questo strumento ho utilizzato molto i file di log generati dalla stessa applicazione, per gli stessi scopi.
	
	\subsection{Analisi dinamica}
	Nell'ultima fase di stage poi, ho svolto col supporto del tutor le attività di validazione, composte dai test di sistema e dal collaudo.
	
	\subsubsection{Test di sistema}
	Per verificare le funzionalità che doveva includere il nuovo prodotto ho effettuato i test di sistema, ovvero una serie di prove al fine di verificare la copertura dei requisiti definiti con le attività di analisi. Questi sono stati svolti in ambiente di collaudo, dove l'integrazione delle componenti dell'applicazione risulta già testata e bisogna testare l'aspetto funzionale, verificando che il prodotto presenti le funzionalità stabilite e che queste non producano errori in fase di esecuzione.\\
	
	Dopo aver compilato il codice e installato il prodotto, quindi, abbiamo testato le funzionalità da me trattate, una a una. Per prima cosa abbiamo stabilito degli scenari di prova per mappare i requisiti e abbiamo stabilito pre-condizioni e post-condizioni. Per ogni attività quindi sono state verificate le post-condizioni, accertando la copertura e il soddisfacimento dei requisiti assegnati ai vari scenari di prova.
	
	\subsubsection{Collaudo}
	Una volta effettuati i test, alcuni consulenti sono entrati in contatto con	il cliente e si sono occupati della validazione dei requisiti funzionali.\\
	
	Per tale attività vengono replicati i test di sistema, nello stesso ambiente di collaudo, per dimostrare ai clienti l'effettivo funzionamento delle \textit{feature} richieste, mostrando loro che i requisiti sono stati soddisfatti.
	
%**************************************************************
\section{Rilascio delle funzionalità}

Nell'ultima settimana di stage, il tutor ha ritenuto corretto che trattassimo anche il rilascio delle nuove \textit{feature} da me sviluppate. Col suo supporto quindi ho utilizzato il sistema di versionamento RTC, mediante Eclipse, per concludere le attività ed associarle alle apposite release unit. Così facendo ho reso disponibili le nuove funzionalità a tutto il team di sviluppo in ambiente di integrazione e, dopo una compilazione, il nuovo software risultava aggiornato.

	\begin{figure}[H]
		\centering
	   	\includegraphics[width=1\textwidth]{immagini/deliver_RTC}
	   	\caption{Interfaccia per eseguire il rilascio delle attività mediante Eclipse e il sistema di versionamento RTC - Fonte: il sito internet di IBM}
	\end{figure}
	
	\begin{figure}[H]
		\centering
	   	\includegraphics[width=0.6\textwidth]{immagini/build_RTC}
	   	\caption{Interfaccia per eseguire la compilazione del software in un determinato ambiente mediante Eclipse e il sistema di versionamento RTC - Fonte: il sito di RTC jazz.net}
	\end{figure}

Dopo le attività di test ho potuto effettuare i rilasci anche in ambiente di collaudo, dove ho svolto la validazione. Successivamente il cliente ha controllato il lavoro svolto, installando poi la nuova versione di ELISE nei server di produzione, per l'utilizzo vero e proprio nelle diverse filiali del gruppo del Banco Popolare.
             % Stage
%% !TEX encoding = UTF-8
% !TEX TS-program = pdflatex
% !TEX root = ../Lahmer_Abdelilah_tesi.tex
% !TEX spellcheck = it-IT

%**************************************************************
\chapter{Valutazione retrospettiva}

\section{Bilancio formativo}
%In questa sezione parlerò del bilancio formativo dopo l'esperienza di stage, ovvero ciò che ho imparato durante tutto il periodo di stage.


\subsection{Conoscenze preliminari}
%In questa sezione parlerò delle conoscenze preliminari che a mio avviso sono indispensabile per lo svolgimento di quella che è stata la mia esperienza di stage.

Per lo svolgimento dello stage non erano richieste da parte dell'azienda particolari conoscenze preliminari teorico-economiche o che riguardassero i linguaggi di programmazione che si sarebbero andati ad utilizzare. Questo proprio perché era prevista la formazione su entrambi i fronti. Lo scopo dello stage, infatti, era formativo.\\

Il Corso di Laurea Triennale in Informatica, nonchè il percorso di Scuola Superiore affrontato, mi hanno fornito una buona formazione nello sviluppo software, permettendomi la comprensione di gran parte delle attività facenti parte del ciclo di sviluppo applicato al progetto aziendale.\\

La mia preparazione sotto l'aspetto tecnico ha aiutato al raggiungimento di discreti risultati nel rispettivo ambito, mentre la quasi nulla preparazione sotto l'aspetto economico è stata un deficit importante per il raggiungimento di buoni risultati nel corrispettivo ambito funzionale; tutto questo causato anche dalla scarsa dedizione alla formazione da parte dell'azienda.\\

A parer mio, quindi, per il raggiungimento dei risulati che l'azienda si aspetta un minimo di base su entrambi i fronti sarebbe indispensabile.

\newpage

\subsection{Conoscenze}
%In questa sottosezione riporterò le conoscenze acquisite con l'esperienza di stage, ovvero ciò che mi è stato insegnato.

A partire dalle conoscenze preliminari in ambito tecnico, tramite il lavoro di stage, ho avuto modo di ampliare le mie conoscenze, apprendendo concetti a me nuovi riguardo le modalità di sviluppo in ambiente \textit{host}.\\

Sempre a livello tecnico ho acquisito conoscenze sulla metodologia di lavoro all'interno di grandi progetti, realizzando realmente il concetto di "Ciclo di Sviluppo".\\

Infine nel corso dello stage ho appreso molte conoscenze di natura economica, e precisamente nell'ambito della finanza bancaria.

\subsection{Abilità}
%In questa sottosezione riporterò le abilità acquisite con l'esperienza di stage, ovvero ciò che ho imparato.

In fatto di abilità acquisite durante il periodo di stage viene prima quella di analisi nel contesto bancario, grazie a questo bimestre infatti ho acquisito capacità e modalità di analisi di funzionalità, e in parte anche tecnica, nell'ambito finanziario, anche se non in modo approfondito.

\subsection{Competenze}
%In questa sottosezione riporterò le competenze acquisite con l'esperienza di stage, ovvero ciò che dopo lo stage SO e SO FARE.

Per quanto riguarda le competenze, invece, lo stage mi ha permesso di acquisire praticità di progettazione all'interno di architetture \textit{host} e sviluppo in linguaggio COBOL\glossario\ di soluzioni orientate al campo della finanza.

%\newpage

\section{Obiettivi raggiunti}
%In questa sezione parlerò degli obiettivi raggiunti, cercando di schematizzare il tutto tramite grafici.

Durante il periodo di stage ho cercato di svolgere le attività necessarie al raggiungimento degli obiettivi prefissati in fase di pianificazione. Le seguenti tabelle elencano i risultati ottenuti.

\subsection{Obbligatori}
%In questa sottosezione riporterò gli obiettivi obbligatori che sono stati raggiunti.

		\begin{center}
		  \bgroup
		  \def\arraystretch{1.4}
		   %\setlength\arrayrulewidth{0.6pt}
		   \begin{longtable}{ | p{9cm} | p{2cm} | }  \hline
			 
			 \cellcolor[gray]{0.9} \textbf{Obiettivi Obbligatori} & \cellcolor[gray]{0.9} \textbf{Esito} \\ \hline
						 
			 Acquisizione di padronanza dell'ambiente di sviluppo Mainframe & Completato \\ \hline
			 Acquisizione di padronanza delle modalità di sviluppo in ambiente Mainframe & Completato \\ \hline
			 Studio e comprensione del linguaggio COBOL & Completato \\ \hline
			 Acquisizione di padronanza di interazione con database DB2 e relativi strumenti & Completato \\ \hline
			 Implementazione di applicazioni di esempio per le funzionalità basilari & Completato \\ \hline
			 Acquisizione di padronanza d'uso di ELISE & Completato \\ \hline
			 Comprensione corretta di analisi tecniche & Completato \\ \hline
			 Integrazione nel team di sviluppo e acquisizione competenze nelle dinamiche di gruppo & Completato \\ \hline
			 Comprensione e acquisizione familiarità con la documentazione di analisi funzionale & Completato \\ \hline
			 Implementazione di modifiche basilari dell'applicazione ambito di progetto & Completato \\ \hline
			
			\caption{Tabella degli obiettivi obbligatori raggiunti}
			
		    \end{longtable}
		  \egroup
		\end{center}


\subsection{Desiderabili}
%In questa sottosezione riporterò gli obiettivi desiderabili che sono stati raggiunti.

	\begin{center}
		  \bgroup
		  \def\arraystretch{1.4}
		   %\setlength\arrayrulewidth{0.6pt}
		   \begin{longtable}{ | p{9cm} | p{2cm} | }  \hline
			 
			 \cellcolor[gray]{0.9} \textbf{Obiettivi Obbligatori} & \cellcolor[gray]{0.9} \textbf{Esito} \\ \hline
						 
			Raggiungimento di un buon livello di autonomia nell'analisi di funzionalità & Completato  \\ \hline
			Raggiungimento di un buon livello di concepimento, anche se parziale, delle modalità di traduzione delle analisi di funzionalità in analisi tecnica & Completato \\ \hline
			Capacità di portare a termine le attività lavorative secondo le tempistiche stabilite & Completato \\ \hline
			Capacità di portare a termine le attività lavorative anche in situazioni critiche & Completato \\ \hline
			Conoscenza delle norme di sicurezza relative all'ambiente di lavoro & Completato \\ \hline
			Comprensione e acquisizione familiarità con concetti teorici in ambito economico & Completato  \\ \hline
			Acquisizione di padronanza delle attrezzature presenti in azienda in funzione del proprio lavoro & Completato \\ \hline
			
			\caption{Tabella degli obiettivi desiderabili raggiunti}
			
		    \end{longtable}
		  \egroup
		\end{center}

\subsection{Facoltativi}
%In questa sottosezione riporterò gli obiettivi facoltativi che sono stati raggiunti.

	\begin{center}
		  \bgroup
		  \def\arraystretch{1.4}
		   %\setlength\arrayrulewidth{0.6pt}
		   \begin{longtable}{ | p{9cm} | p{2cm} | }  \hline
			 
			 \cellcolor[gray]{0.9} \textbf{Obiettivi Obbligatori} & \cellcolor[gray]{0.9} \textbf{Esito} \\ \hline
						 
			Studio delle meccaniche di comunicazione con la parte \textit{web} (\textit{front-end}) dell'applicazione ELISE & Non Completato  \\ \hline
			Rilascio di nuova funzionalità analizzata e sviluppata  & Non Completato  \\ \hline

			
			\caption{Tabella degli obiettivi desiderabili raggiunti}
			
		    \end{longtable}
		  \egroup
		\end{center}

\subsection{Obiettivi personali}
%In questa sottosezione riporterò gli obiettivi personali che sono stati raggiunti.
Inizialmente con l'inizio del percorso di stage mi ero posto degli obiettivi, tali obiettivi sono stati parzialmente raggiunti. Con la soddisfacente richiesta dell'azienda di proseguire con i restanti quattro mesi di stage, infatti, ho raggiunto l'obiettivo più grande, ovvero proseguire sulla strada che portava verso la concretizzazione della mia ambizione.\\

Oltre a questo al termine del periodo di stage ero in grado di contribuire lato tecnico e potevo parzialmente dare un contributo lato funzionale, su cui continuava il mio processo di apprendimento.

\section{Gap analysis}
In questa sezione riporterò quelli che secondo me sono punti importanti che nel percorso accademico non vengono trattati ma che nel mondo del lavoro sono essenziali o dati per scontato.             % Valutazioni

%**************************************************************
% Materiale finale
%**************************************************************
\appendix 
%aggiunto per fix glossario                              
\glsaddall
\printglossaries
\backmatter
%% !TEX encoding = UTF-8
% !TEX TS-program = pdflatex
% !TEX root = ../Lahmer_Abdelilah_tesi.tex
% !TEX spellcheck = it-IT

%**************************************************************
% Ringraziamenti
%**************************************************************
\cleardoublepage
\phantomsection
\pdfbookmark{Ringraziamenti}{ringraziamenti}

\leavevmode	\newline

\begin{flushright}{
	\slshape    
	``He who does not thank people, does not thank God''} \\ 
	\medskip
    --- Prophet Muhammad (PBUH)
\end{flushright}


\bigskip

\begingroup
\let\clearpage\relax
\let\cleardoublepage\relax
\let\cleardoublepage\relax

\chapter*{Ringraziamenti}

 \noindent \textit{Innanzitutto vorrei ringraziare i miei genitori, Najeh e Latifa, per avermi accompagnato e concesso di arrivare fin qui. Grazie inoltre alla mia intera famiglia per il sostegno e per essermi sempre stati vicini.}\\

\noindent \textit{Ringrazio i compagni di studi per tutti i bellissimi anni passati insieme, in particolare i colleghi di Answer Group.}\\

\noindent \textit{La più sentita gratitudine inoltre agli amici più stretti, in particolare ringrazio Hamza, Abdourahmane, Amir, Mustafa e Sara per tutto il loro affetto e sostegno ricevuto.}\\

%\noindent \textit{Ringrazio Sopra Steria Group S.p.A. e tutti i dipendenti della sede di Padova per avermi accolto e seguito durante il tirocinio.}\\

\noindent \textit{Ringrazio sentitamente, infine, il prof. \myProf, relatore della mia Tesi, per l'aiuto, i preziosi consigli e la pazienza che mi ha dedicato per lo svolgimento del lavoro.}\\

\bigskip

\noindent\textit{\myLocation, \myTime}
\hfill \myName

\endgroup

%\blankpage
%\blankpage


\end{document}

$
The book class has some advantages over the report class since it defines three
commands (\frontmatter, \mainmatter, and \backmatter) that control the page
number and chapter numbering formats. In the frontmatter, pages are numbered
with lower case Roman numbers (i, ii, iii, etc.) and the chapters are not numbered
(as if the asterisk version /chapter*{} was used). In the mainmatter, pages are
numbered with Arabic numbers (the numbers start from 1) and the chapters
are numbered with Arabic numbers as well. In the backmatter, the pages are
numbered as in the mainmatter (numbering continues) but the chapters are not
numbered.
$
