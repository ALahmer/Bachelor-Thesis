% !TEX encoding = UTF-8
% !TEX TS-program = pdflatex
% !TEX root = ../Lahmer_Abdelilah_tesi.tex
% !TEX spellcheck = it-IT

%**************************************************************
% Sommario
%**************************************************************
%\cleardoublepage
\phantomsection
\pdfbookmark{Convenzioni tipografiche}{Convenzioni tipografiche}
\begingroup
\let\clearpage\relax
\let\cleardoublepage\relax
\let\cleardoublepage\relax

%\chapter*{Convenzioni tipografiche}
										%\chapter{Convenzioni tipografiche}
%\section*{Convenzioni tipografiche}

\leavevmode	\newline
\leavevmode	\newline
\leavevmode	\newline
\begin{Huge}
Convenzioni tipografiche
\end{Huge}
\leavevmode	\newline

Per la stesura del documento ho adottato le seguenti norme tipografiche:\\

\begin{itemize}
	\item L'utilizzo del \textit{corsivo} per le parole di ambito tecnico;
	\item L'utilizzo del \textit{corsivo} per i termini in lingua inglese che non dispongono di un corrispettivo termine in italiano, o che nel contesto in cui vengono utilizzati sia meglio adoperare il termine inglese; 
	\item L'indicazione con una G a pedice della prima occorrenza del capitolo di tutti i termini che necessitano di una spiegazione esplicita, definita nel glossario presente a fine documento;
	\item L'utilizzo del \textbf{grassetto} per evidenziare termini di rilievo nei paragrafi.
\end{itemize}


%\vfill
%
%\selectlanguage{english}
%\pdfbookmark{Abstract}{Abstract}
%\chapter*{Abstract}
%
%\selectlanguage{italian}

\endgroup			

%\vfill

