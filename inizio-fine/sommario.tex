% !TEX encoding = UTF-8
% !TEX TS-program = pdflatex
% !TEX root = ../Lahmer_Abdelilah_tesi.tex
% !TEX spellcheck = it-IT

%**************************************************************
% Sommario
%**************************************************************
\cleardoublepage
\phantomsection
\pdfbookmark{Sommario}{Sommario}
\begingroup
\let\clearpage\relax
\let\cleardoublepage\relax
\let\cleardoublepage\relax

\chapter*{Sommario}

Il presente documento riassume il lavoro svolto durante il periodo di stage, della durata di circa 300 ore, presso l’azienda Sopra Steria Group S.p.A con sede a Padova.
%La mission di Sopra Steria Group consiste nell'accompagnare e aiutare i suoi clienti a conseguire il successo attraverso il processo di trasformazione dei loro processi di business e dei loro sistemi informativi.
%Io sono stato inserito nella Divisione Servizi Finanziari, sezione che si occupa di sviluppo e manutenzione di sistemi bancari.
%Il tirocinio formativo e orientativo che mi è stato proposto ha avuto lo scopo di avviarmi verso la conoscenza della realtà lavorativa, approfondendo e verificando l'apprendimento ricevuto nel percorso degli studi con un'esperienza soggettiva legata direttamente alla realtà economica e produttiva del territorio, svolta nell'ambito di una realtà multinazionale.
%Nello specifico sono stato inserito in un gruppo di lavoro che opera su vari progetti di analisi e sviluppo, nonché manutenzione, di soluzioni bancarie per primari istituti di credito sul territorio italiano, quali Banca Popolare di Verona e Banco Popolare di Milano.
L’obiettivo minimo del tirocinio è stato l'acquisire padronanza dell'ambiente di sviluppo MAINFRAME, del linguaggio di programmazione COBOL e l'essere in grado di comprendere correttamente le analisi tecniche.
Obiettivo desiderabile è stato raggiungere anche una discreta autonomia nell'analisi di funzionalità e il concepimento, anche se parziale, di modalità di traduzione di queste in analisi tecnica.

%\vfill
%
%\selectlanguage{english}
%\pdfbookmark{Abstract}{Abstract}
%\chapter*{Abstract}
%
%\selectlanguage{italian}

\endgroup			

\vfill

