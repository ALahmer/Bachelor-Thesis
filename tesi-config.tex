%**************************************************************
% file contenente le impostazioni della tesi
%**************************************************************

%**************************************************************
% Frontespizio
%**************************************************************
\newcommand{\myName}{Abdelilah Lahmer}                         % autore
\newcommand{\myTitle}{Meccanismi di programmazione back-end e analisi in ambito bancario}                    
%\newcommand{\myTitle}{Meccanismi di programmazione back-end e analisi $funzionale $in ambito bancario}                    
\newcommand{\myDegree}{Tesi di laurea triennale}                % tipo di tesi
\newcommand{\myUni}{Università degli Studi di Padova}           % università
\newcommand{\myFaculty}{Corso di Laurea in Informatica}         % facoltà
\newcommand{\myDepartment}{Dipartimento di Matematica \\ "Tullio Levi-Civita"}          % dipartimento
\newcommand{\myProf}{Tullio Vardanega}                          % relatore
\newcommand{\myLocation}{Padova}                               % dove
\newcommand{\myAA}{Dicembre 2017}                                   % anno accademico
\newcommand{\myTime}{Dicembre 2017}                             % quando


%**************************************************************
% Impostazioni di impaginazione
% see: http://wwwcdf.pd.infn.it/AppuntiLinux/a2547.htm
%**************************************************************

\setlength{\parindent}{0pt}   		% larghezza rientro della prima riga
\setlength{\parskip}{0pt}   		% distanza tra i paragrafi

%\geometry{top=3cm, bottom=4cm, left=3.6cm, right=3.6cm}		% margini copertina
\geometry{top=4cm, bottom=5cm, left=4cm, right=4cm}			% margini tesi

\pagestyle{fancy}					% tipo di impaginazione header e footer
\renewcommand{\chaptermark}[1]{%
	\markboth{\MakeUppercase{\thechapter.\ #1}}{}
}
\renewcommand{\sectionmark}[1]{}
\fancyhead[LE,RO]{\textsl{\leftmark}}
\fancyhead[LO,RE]{\textsl{\rightmark}}
\fancyfoot[C]{\vspace*{1\baselineskip}\thepage}	% prints the page number on the center-bottom side of the footer

\fancypagestyle{plain}{
	\fancyhf{}
  	\fancyfoot[C]{\vspace*{1\baselineskip}\thepage}	% prints the page number on the center-bottom side of the footer
  	\renewcommand{\headrulewidth}{0pt}% Line at the header invisible
  	\renewcommand{\footrulewidth}{0pt}% Line at the footer visible
}

\newcommand\blankpage{%
	\newpage
    \null
    \thispagestyle{empty}%
    \addtocounter{page}{-1}%
    \newpage}

%**************************************************************
% Impostazioni di biblatex
%**************************************************************
\bibliography{bibliografia} % database di biblatex 

\defbibheading{bibliography}
{
    \cleardoublepage
    \phantomsection 
    \addcontentsline{toc}{chapter}{\bibname}
    \chapter*{\bibname\markboth{\bibname}{\bibname}}
}

\setlength\bibitemsep{1.5\itemsep} % spazio tra entry

\DeclareBibliographyCategory{opere}
\DeclareBibliographyCategory{web}

\addtocategory{opere}{womak:lean-thinking}
\addtocategory{web}{site:agile-manifesto}

\defbibheading{opere}{\section*{Riferimenti bibliografici}}
\defbibheading{web}{\section*{Siti Web consultati}}


%**************************************************************
% Impostazioni di caption
%**************************************************************
\captionsetup{
    tableposition=top,
    figureposition=bottom,
    font=small,
    format=hang
}

\captionsetup[figure]{
	name=Figura
}

\captionsetup[table]{
	name=Tabella
}

%**************************************************************
% Impostazioni di glossaries
%**************************************************************

%**************************************************************
% Acronimi
%**************************************************************
%\renewcommand{\acronymname}{Acronimi e abbreviazioni}
%
%\newacronym[description={\glslink{apig}{Application Program Interface}}]
%    {api}{API}{Application Program Interface}
%
%\newacronym[description={\glslink{umlg}{Unified Modeling Language}}]
%    {uml}{UML}{Unified Modeling Language}

%**************************************************************
% Glossario
%**************************************************************
%\renewcommand{\glossaryname}{Glossario}

\newglossaryentry{ajax}
{
    name=AJAX,
    text=Asynchronous JavaScript and XML,
    sort=ajax,
    description={Asynchronous JavaScript and XML è una tecnica di sviluppo software per la realizzazione di applicazioni web che interagiscano in background con il server senza bisogno di ricaricare la pagina nel browser}
}

\newglossaryentry{batch}
{
    name=BATCH,
    text=BATCH,
    sort=batch,
    description={L'esecuzione non immediata, ma rimandata nel tempo di programmi}
}

\newglossaryentry{cics}
{
    name=CICS,
    text=Customer Information Control System,
    sort=cics,
    description={Customer Information Control System è una famiglia di application server che fornisce la gestione di transazioni online e connettività per applicazioni su mainframe IBM}
}

\newglossaryentry{cobol}
{
    name=COBOL,
    text=COmmon Business-Oriented Language,
    sort=cobol,
    description={COmmon Business-Oriented Language è uno dei primi linguaggi di programmazione ad essere stato sviluppato. Nonostante sia un linguaggio datato, il COBOL è tuttora presente in molte applicazioni software commerciali di tipo bancario, specie lato mainframe (es. CICS), che non si è preferito o voluto migrare in altra tecnologia software}
}

\newglossaryentry{covenant}
{
    name=Covenant,
    text=Covenant,
    sort=covenant,
    description={In finanza con il termine covenant si indica un accordo che intercorre tra un'impresa e i suoi finanziatori, includendo determinate clausole e parametri per tutelarli}
}

\newglossaryentry{ctg}
{
    name=CTG,
    text=CICS Transaction Gateway,
    sort=ctg,
    description={CICS Transaction Gateway offre un accesso sicuro a sistemi CICS da applicazioni Java, Java EE, .NET, C e C++ usando i protocolli internet}
}
 
\newglossaryentry{dbms}
{
    name=DBMS,
    text=Database Management System,
    sort=dbms,
    description={Database Management System è un sistema di gestione di basi di dati, consente la creazione, la manipolazione e l’interrogazione efficiente di database}
}

\newglossaryentry{eis}
{
    name=EIS,
    text=Enterprise Information Systems,
    sort=eis,
    description={Enterprise Information Systems, sistemi informativi d'impresa per migliorare le funzionalità dei processi di business delle aziende rapportandosi con grandi quantità di dati e instaurando un sistema di gestione centrale delle informazioni}
}

\newglossaryentry{front-end}
{
    name=Fornt-end,
    text=Fornt-end,
    sort=front-end,
    description={denotazione della parte del software visibile all'utente, spesso un'interfaccia grafica con cui egli può interagire ed usufruire delle funzionalità offerte}
}


\newglossaryentry{iframe}
{
    name=iframe,
    text=iframe,
    sort=iframe,
    description={inline frame è usato per l'inclusione di una finestra su un'altra risorsa nella pagina corrente}
}

\newglossaryentry{ict}
{
    name=ICT,
    text=Information and Communications Technology,
    sort=ict,
    description={Information and Communications Technology, ovvero le tecnologie dell’informazione e della comunicazione, 
spesso questo acronimo è usato per definire l'ambito di esercizio delle attività di un'azienda}
}

\newglossaryentry{jar}
{
    name=JAR,
    text=Java Archive,
    sort=jar,
    description={Java Archive è un archivio che raccoglie in un unico file le classi Java di un programma e le relative risorse, per distribuire il software in una piattaforma compatibile}
}

\newglossaryentry{javabeans}
{
    name=JavaBeans,
    text=JavaBeans,
    sort=javabeans,
    description={è una particolare convenzione per scrivere classi Java in modo da includere più oggetti in uno solo, permettendo la serializzazione e l'interscambio dell'intero insieme o l'impostazione e la fruizione dei singoli attributi}
}

\newglossaryentry{jca}
{
    name=JCA,
    text=Java EE Connector Architecture,
    sort=jca,
    description={Java EE Connector Architecture è una tecnologia basata su linguaggio Java per la connessione di application server e EIS come parte dell'applicazione d'impresa}
}

\newglossaryentry{jcl}
{
    name=JCL,
    text=Job Control Language,
    sort=jcl,
    description={In informatica il Job Control Language (JCL) è un linguaggio di scripting utilizzato nei sistemi operativi IBM DOS/VSE, OS/VS1 ed MVS per eseguire (in gergo lanciare) una procedura batch su un sistema generalmente mainframe}
}

\newglossaryentry{jdbc}
{
    name=JDBC,
    text=Java DataBase Connectivity,
    sort=jdbc,
    description={Java DataBase Connectivity è una tecnologia usata per connettere applicazioni Java EE ai diversi database per l'accesso e la gestione della persistenza dei dati}
}


\newglossaryentry{jndi}
{
    name=JNDI,
    text=Java Naming and Directory Interface,
    sort=jndi,
    description={Java Naming and Directory Interface è un servizio offerto da Java per ottenere dati e oggetti tramite un nome, che possono essere memorizzati su un server, su file o su database}
}

\newglossaryentry{mvc}
{
    name=MVC,
    text=Model View Controller,
    sort=mvc,
    description={Model View Controller, un pattern architetturale molto diffuso nello sviluppo di sistemi software, in particolare nell’ambito della programmazione ad oggetti}
}

\newglossaryentry{reflection}
{
    name=Reflection,
    text=Reflection,
    sort=reflection,
    description={è una tecnica informatica con cui un programma puo esaminare e modificare la propria struttura e il suo comportamento durante l'esecuzione}
}

\newglossaryentry{servlet}
{
    name=Servlet,
    text=Servlet,
    sort=servlet,
    description={è un particolare tipo di classe Java fornito dalla versione Java EE per estendere le funzionalità di un server, implementando nelle applicazioni web la controparte	di altre tecnologie per contenuti web dinamici}
} % database di termini
\makeglossaries


%**************************************************************
% Impostazioni di graphicx
%**************************************************************
\graphicspath{{immagini/}} % cartella dove sono riposte le immagini


%**************************************************************
% Impostazioni di hyperref
%**************************************************************
\hypersetup{
    %hyperfootnotes=false,
    %pdfpagelabels,
    %draft,	% = elimina tutti i link (utile per stampe in bianco e nero)
    colorlinks=true,
    linktocpage=true,
    pdfstartpage=1,
    pdfstartview=FitV,
    % decommenta la riga seguente per avere link in nero (per esempio per la stampa in bianco e nero)
    colorlinks=false, linktocpage=false, pdfborder={0 0 0}, pdfstartpage=1, pdfstartview=FitV,
    breaklinks=true,
    pdfpagemode=UseNone,
    pageanchor=true,
    pdfpagemode=UseOutlines,
    plainpages=false,
    bookmarksnumbered,
    bookmarksopen=true,
    bookmarksopenlevel=1,
    hypertexnames=true,
    pdfhighlight=/O,
    %nesting=true,
    %frenchlinks,
    urlcolor=webbrown,
    linkcolor=RoyalBlue,
    citecolor=webgreen,
    %pagecolor=RoyalBlue,
    %urlcolor=Black, linkcolor=Black, citecolor=Black, %pagecolor=Black,
    pdftitle={\myTitle},
    pdfauthor={\textcopyright\ \myName, \myUni, \myFaculty},
    pdfsubject={},
    pdfkeywords={},
    pdfcreator={pdfLaTeX},
    pdfproducer={LaTeX}
}

%**************************************************************
% Impostazioni di itemize
%**************************************************************
%\renewcommand{\labelitemi}{$\bullet$}

%\renewcommand{\labelitemi}{$\bullet$}
%\renewcommand{\labelitemii}{$\cdot$}
%\renewcommand{\labelitemiii}{$\diamond$}
%\renewcommand{\labelitemiv}{$\ast$}


%**************************************************************
% Impostazioni di listings
%**************************************************************
\lstset{
    language=[LaTeX]Tex,%C++,
    keywordstyle=\color{RoyalBlue}, %\bfseries,
    basicstyle=\small\ttfamily,
    %identifierstyle=\color{NavyBlue},
    commentstyle=\color{Green}\ttfamily,
    stringstyle=\rmfamily,
    numbers=none, %left,%
    numberstyle=\scriptsize, %\tiny
    stepnumber=5,
    numbersep=8pt,
    showstringspaces=false,
    breaklines=true,
    frameround=ftff,
    frame=single
} 


%**************************************************************
% Impostazioni di xcolor
%**************************************************************
\definecolor{webgreen}{rgb}{0,.5,0}
\definecolor{webbrown}{rgb}{.6,0,0}
\definecolor{Gray}{gray}{0.85}


%**************************************************************
% Altro
%**************************************************************

\newcommand{\omissis}{[\dots\negthinspace]} % produce [...]

% eccezioni all'algoritmo di sillabazione
\hyphenation
{
    ma-cro-istru-zio-ne
    gi-ral-din
}

\newcommand{\sectionname}{sezione}
\addto\captionsitalian{\renewcommand{\figurename}{figura}
                       \renewcommand{\tablename}{tabella}}

\newcommand{\glossario}{\textsubscript{\textit{G}}}

\newcommand{\intro}[1]{\emph{\textsf{#1}}}

%**************************************************************
% Environment per ``rischi''
%**************************************************************
\newcounter{riskcounter}                % define a counter
\setcounter{riskcounter}{0}             % set the counter to some initial value

%%%% Parameters
% #1: Title
\newenvironment{risk}[1]{
    \refstepcounter{riskcounter}        % increment counter
    \par \noindent                      % start new paragraph
    \textbf{\arabic{riskcounter}. #1}   % display the title before the 
                                        % content of the environment is displayed 
}{
    \par\medskip
}

\newcommand{\riskname}{Rischio}

\newcommand{\riskdescription}[1]{\textbf{\\Descrizione:} #1.}

\newcommand{\risksolution}[1]{\textbf{\\Soluzione:} #1.}

%**************************************************************
% Environment per ``use case''
%**************************************************************
\newcounter{usecasecounter}             % define a counter
\setcounter{usecasecounter}{0}          % set the counter to some initial value

%%%% Parameters
% #1: ID
% #2: Nome
\newenvironment{usecase}[2]{
    \renewcommand{\theusecasecounter}{\usecasename #1}  % this is where the display of 
                                                        % the counter is overwritten/modified
    \refstepcounter{usecasecounter}             % increment counter
    \vspace{10pt}
    \par \noindent                              % start new paragraph
    {\large \textbf{\usecasename #1: #2}}       % display the title before the 
                                                % content of the environment is displayed 
    \medskip
}{
    \medskip
}

\newcommand{\usecasename}{UC}

\newcommand{\usecaseactors}[1]{\textbf{\\Attori Principali:} #1. \vspace{4pt}}
\newcommand{\usecasepre}[1]{\textbf{\\Precondizioni:} #1. \vspace{4pt}}
\newcommand{\usecasedesc}[1]{\textbf{\\Descrizione:} #1. \vspace{4pt}}
\newcommand{\usecasepost}[1]{\textbf{\\Postcondizioni:} #1. \vspace{4pt}}
\newcommand{\usecasealt}[1]{\textbf{\\Scenario Alternativo:} #1. \vspace{4pt}}

%**************************************************************
% Environment per ``namespace description''
%**************************************************************

\newenvironment{namespacedesc}{
    \vspace{10pt}
    \par \noindent                              % start new paragraph
    \begin{description} 
}{
    \end{description}
    \medskip
}

\newcommand{\classdesc}[2]{\item[\textbf{#1:}] #2}

%**************************************************************
% Note piè di pagina
%**************************************************************

%\renewcommand*{\footnoterule}{ \kern -1pt \hrule width \columnwidth height 1pt \kern 1pt}