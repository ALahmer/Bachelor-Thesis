% !TEX encoding = UTF-8
% !TEX TS-program = pdflatex
% !TEX root = ../Lahmer_Abdelilah_tesi.tex
% !TEX spellcheck = it-IT

%**************************************************************
\chapter{Valutazione retrospettiva}
\label{cap:valutazione-retrospettiva}
%**************************************************************

%\intro{Breve introduzione al capitolo}\\

%**************************************************************
\section{Conoscenze preliminari}

Per lo svolgimento del lavoro di stage non era richiesta una conoscenza approfondita dei linguaggi che si andavano ad utilizzare, proprio perché era prevista una prima fase di studio delle tecnologie e lo scopo dello stage era formativo, in modo da arricchire le mie competenze curricolari.\\

Nonostante ciò, la mia preparazione riguardo le tecnologie web,	maturata nel corso di studi, è stata molto utile nel comprendere i nuovi aspetti che andavo a trattare in azienda, consolidando tali conoscenze. L'aspetto che ha richiesto maggiore impegno da parte mia è stata l'analisi del progetto ELISE e le prime fasi di approccio ad esso, nel tentativo di individuare percorsi a me conosciuti e comprendere al meglio i suoi funzionamenti, essendo un applicativo molto grande ed intricato nei suoi processi.\\

Il Corso di Laurea triennale in Informatica mi ha fornito una buona formazione nello sviluppo software, permettendomi la comprensione del ciclo di sviluppo applicato al progetto richiesto nel mio periodo di stage, senza troppe difficoltà.\\
	
Questi aspetti preliminari hanno permesso che il progetto di stage non subisse imprevisti o interruzioni dovute a mie lacune, concentrandomi per più tempo sugli studi e le applicazioni previste dal piano di lavoro.

%**************************************************************
\newpage
\section{Obiettivi raggiunti}

Durante il lavoro di stage ho svolto le attività necessarie al raggiungimento degli obiettivi prefissati in fase di pianificazione.
Le seguenti tabelle raccolgono i risultati ottenuti per ciascun obbiettivo, suddividendoli per la loro categoria.

\subsubsection{Obbligatori}

	\begin{table}[H]
		\def\arraystretch{1.2}
		\begin{tabular}{ | p{10cm} | p{2cm} | }
		
		\rowcolor{Gray}
		\hline \textbf{Obiettivo} & \textbf{Esito} \\ \hline
		
		Studio e acquisizione di padronanza dell'ambiente di sviluppo Eclipse & Completato \\ \hline
		Studio e comprensione della piattaforma Java EE & Completato \\ \hline
		Installazione e utilizzo di diversi \textit{application server} e diverse JVM & Completato \\ \hline
		Acquisizione familiarità con la programmazione web in ambito Java (tecnologie JSP, JSTL, Servlet\glossario ) & Completato \\ \hline
		Acquisizione tecniche di programmazione con framework Struts & Completato \\ \hline
		Studio e utilizzo del sistema di versionamento RTC & Completato \\ \hline
		Studio e utilizzo della tecnologia AJAX\glossario & Completato \\ \hline
		Implementazione di applicazioni di esempio per le funzionalità basilari & Completato \\ \hline
		Analisi di una complessa applicazione reale in architettura Java EE & Completato \\ \hline
		Integrazione nel team di sviluppo e acquisizione competenze nelle dinamiche di gruppo & Completato \\ \hline
		Comprensione e acquisizione familiarità con la documentazione di analisi e specifica delle attività & Completato \\ \hline
		Implementazione di modifiche basilari dell'applicazione ambito di progetto & Completato \\ \hline
		Implementazione di modifiche complesse dell'applicazione ambito di progetto & Completato \\ \hline
		
		\end{tabular}
		\vspace{1mm}
		\caption{Tabella degli obiettivi obbligatori raggiunti durante il lavoro di stage}
	\end{table}


\subsubsection{Desiderabili}

	\begin{table}[H]
		\def\arraystretch{1.2}
		\begin{tabular}{ | p{10cm} | p{2cm} | }
		
		\rowcolor{Gray}
		\hline \textbf{Obiettivo} & \textbf{Esito} \\ \hline
		
		Raggiungimento di un buon livello di autonomia nell'utilizzo di Java EE & Completato \\ \hline
		Raggiungimento di un buon livello di autonomia nell'utilizzo di tecnologie web standard (HTML, CSS, JavaScript) & Completato \\ \hline
		Raggiungimento di un buon livello di autonomia nell'utilizzo di tecnologie web in ambito Java (JSP, JSTL, Servlet\glossario ) & Completato \\ \hline
		Acquisizione tecniche di programmazione con framework alternativi (Maven e Hibernate) & Completato \\ \hline
		Studio e utilizzo di un sistema di versionamento alternativo (SVN) & Completato \\ \hline
		Studio e utilizzo della libreria per le stampe PDF, iText & Completato \\ \hline
		Studio e utilizzo della libreria per il \textit{logging}, Log4J & Completato \\ \hline
		Capacità di portare a termine le attività lavorative secondo le tempistiche stabilite, anche in situazioni critiche & Completato \\ \hline
		Conoscenza delle norme di sicurezza relative all'ambiente di lavoro & Completato \\ \hline
		
		\end{tabular}
		\vspace{1mm}
		\caption{Tabella degli obiettivi desiderabili raggiunti durante il lavoro di stage}
	\end{table}


\subsubsection{Facoltativi}

	\begin{table}[H]
		\def\arraystretch{1.2}
		\begin{tabular}{ | p{10cm} | p{2cm} | }
		
		\rowcolor{Gray}
		\hline \textbf{Obiettivo} & \textbf{Esito} \\ \hline
		
		Studio delle meccaniche di comunicazione con l'area di business per la gestione dei dati & Completato \\ \hline
		Partecipazione alle attività di test di integrazione dell'applicazione ambito di progetto & Non \newline completato \\ \hline
		Partecipazione alle attività di collaudo dell'applicazione ambito di progetto & Completato \\ \hline
		Rilascio delle nuove funzionalità sviluppate nelle attività assegnate & Completato \\ \hline
		
		\end{tabular}
		\vspace{1mm}
		\caption{Tabella degli obiettivi facoltativi raggiunti durante il lavoro di stage}
	\end{table}

Si denota quindi un completamento globale degli obiettivi pianificati, ad unica eccezione delle attività di integrazione software, in quanto queste sono state assegnate ad altri team di sviluppo.
%Durante il mio lavoro infatti si succedevano molteplici dinamismi all'interno dell'azienda e io e il tutor non abbiamo potuto far fronte a tutti. \\

%**************************************************************
\newpage
\section{Bilancio formativo}

A partire dalle mie conoscenze preliminari nell'ambito ricoperto dal progetto ho avuto modo, tramite il lavoro di stage, di ampliare le mie conoscenze e spingermi verso concetti a me nuovi riguardo la progettazione architetturale nel web. Grazie alle spiegazioni del tutor aziendale, a cui devo la facilità con cui si è svolto il progetto, ho potuto ricevere molte nozioni, anche non approfondite ma che di volta in volta mi abilitavano a nuove possibilità di implementazione, inglobandomi in un ambiente stimolante e dinamico.\\

Tutto ciò mi ha permesso di imparare molto, arricchendo le mie conoscenze nello sviluppo web, utilizzando software di cui non ero pratico e consolidando le mie conoscenze a livello teorico, applicandole nella realtà.\\

Ho guadagnato padronanza degli sviluppi web mediante Java EE e le tecnologie incluse in tale piattaforma, come JSP e JSTL che non avevo mai utilizzato. Ho interagito con sistemi diversi di versionamento, ho imparato nuove tecniche di programmazione, come AJAX\glossario , la Reflection\glossario\ e le \textit{properties}, di cui non ero a conoscenza o avevo ricevuto solo alcuni accenni all'Università. Ho anche ottenuto una buona padronanza con un nuovo ambiente di sviluppo e diverse librerie molto utili, ad esempio per il \textit{logging}.\\

Grazie allo stage in Sopra Steria ho avuto modo di crescere professionalmente e vivere personalmente un'esperienza lavorativa che ha avuto luogo in un ambiente molto dinamico e stimolante. Ho interagito con molti colleghi anche in altre sedi e ho avuto una visione generale dei processi di un'azienda rilevante nel settore ICT\glossario .\\

Presumibilmente il Corso di Laurea triennale in Informatica è ancora in fase di rodaggio, dopo il passaggio a semestri, si è sentita la pesantezza del carico distribuito	male nei vari insegnamenti, specie riguardo le propedeuticità che spesso in questo corso rappresentano uno scoglio per gli studenti.\\

Ho apprezzato che il corso di Ingegneria del Software abbia dato luogo a diversi seminari tecnoloici e abbia incentrato i progetti presentati nell'uso di moderni linguaggi e paradigmi di programmazione, al contrario degli altri corsi che rimangono piuttosto canonici e trattano poche innovazioni negli ambiti di studio.

% Conclusioni
\newpage
Oltre al successo nel soddisfare gli obiettivi prefissati per la valutazione positiva dello stage, a livello personale ho avuto molta soddisfazione nel raggiungere anche gli obiettivi	personali che mi ero posto nella ricerca dello stage.\\

Ho lavorato in un ambiente professionale e collaborativo, applicando le mie competenze per scopi pratici e utili al team di sviluppo. Ho avuto modo di arricchire il mio profilo tecnico e lavorativo diventando un potenziale interesse ancora maggiore verso le aziende ICT\glossario , come la stessa Sopra Steria che mi ha ospitato.\\

	Sono rimasto soddisfatto quindi dell'esperienza fatta, ma anche della formazione guadagnata presso l'Università nel Corso di Laurea triennale in informatica, stabilendo una solida base di studi su cui costruire la mia carriera.\\
	
	Mi è stata data la possibilità di proseguire il mio percorso di stage in vista di una futura assunzione, dopo il conseguimento di un totale di sei mesi di tirocinio, compresi i due appena svolti. Pertanto ho intenzione di procedere in tale direzione, vista anche la soddisfazione da ambo le parti dopo questa esperienza.

	
%\gls{ajax}
%
%\gls{cics}
%
%\gls{cobol}
%
%\gls{covenant}
%
%\gls{ctg}
% 
%\gls{dbms}
%\gls{eis}
%
%\gls{front-end}
%
%\gls{iframe}
%
%\gls{ict}
%
%\gls{jar}
%
%\gls{javabeans}
%
%\gls{jca}
%
%\gls{jdbc}
%
%\gls{jndi}
%
%\gls{mvc}
%
%\gls{reflection}
%
%\gls{servlet}


%**************************************************************
%\section{Conclusioni}
%Valutazione retrospettiva dell'esperienza di stage.
