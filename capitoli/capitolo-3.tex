% !TEX encoding = UTF-8
% !TEX TS-program = pdflatex
% !TEX root = ../Lahmer_Abdelilah_tesi.tex
% !TEX spellcheck = it-IT

%**************************************************************
\chapter{Il progetto di Stage}

\section{Pianificazione del lavoro}
%In questo inizio di sezione parlerò della pianificazione del lavoro per il progetto di stage in fatto di tempistiche.
Per raggiungere gli obiettivi pianificati nel piano di stage e rispettare i requisiti minimi imposti dall'Università, io e il tutor aziendale abbiamo previsto 304 ore di lavoro, distribuite in circa 8 settimane da 40 ore ciascuna. Ho iniziato lo stage il 17 Maggio 2017 e terminato il 10 Luglio 2017, rimanendo in linea con quanto pianificato inizialmente, senza incorrere in particolari scostamenti.

\subsection{Definizione del piano di lavoro}
In questa sottosezione parlerò della pianificazione del lavoro per il progetto di stage riportando i dati dal piano di lavoro con la programmazione delle ore da dedicare a ciascuna attività.

\newpage

		\begin{center}
		  \bgroup
		  \def\arraystretch{1.4}
		   \setlength\arrayrulewidth{0.6pt}
		   \begin{longtable}{ | p{3cm} | p{9cm} |} \hline
		   %\begin{longtable}{ | p{2.5cm} | p{0.5cm} | p{9cm} |} \hline
		   
		    %\cellcolor[gray]{0.9} \textbf{Durata in ore} & \cellcolor[gray]{0.9} & \cellcolor[gray]{0.9} \textbf{Descrizione dell'attività} \\ \hline
		    
		    \cellcolor[gray]{0.9} \textbf{Durata in ore} &  \cellcolor[gray]{0.9} \textbf{Descrizione dell'attività} \\ \hline


			76 	& FASE 1: FORMAZIONE TECNICA \\ \hline

			\tab \tab 24 & Formazione ambiente di sviluppo Mainframe\\
			\cline{2-2}		%\cline{2-2}
			\tab \tab 52 & Formazione linguaggio di programmazione COBOL\glossario \\	\hline
			
			76 	& FASE 2: FORMAZIONE FUNZIONALE \\ \hline

			\tab \tab 12 &  Formazione sul sistema sviluppato dall'azienda\\
			\cline{2-2}		%\cline{2-2}
			\tab \tab 12 &   Formazione sulla parte relativa al concetto di “Pool”\\
			\cline{2-2}		%\cline{2-2}
			\tab \tab 52 &  Formazione sulla modalità di trasformazione dei concetti funzionali in tecnici\\	\hline			
			
			152 & FASE 3: ANALISI TECNICA E FUNZIONALE MODULO “POOL” \\ \hline

			\tab \tab 40 &  Analisi modulo \\
			\cline{2-2}		%\cline{2-2}
			\tab \tab 40 &  Progettazione dell'analisi funzionale \\
			\cline{2-2}		%\cline{2-2}
			\tab \tab 72 &   Redazione analisi Funzionale e Tecnica modulo \\	\hline			

%			76 &	& FASE 1: FORMAZIONE TECNICA \\ \hline
%
%			& 24 & Formazione ambiente di sviluppo Mainframe\\
%			\cline{2-2}		%\cline{2-2}
%			& 52 & Formazione linguaggio di programmazione COBOL\\	\hline
%			
%			76 	&	& FASE 2: FORMAZIONE FUNZIONALE \\ \hline
%			152 & 	& FASE 3: ANALISI TECNICA E FUNZIONALE MODULO “POOL” \\ \hline

			
			\caption{Pianificazione delle attività di stage}
			
		    \end{longtable}
		  \egroup
		\end{center}
	
\subsection{Livello di autonomia}
In questa sottosezione parlerò del livello di autonomia con cui ho svolto lo stage.

\section{Formazione}
In questo inizio di sezione farò una panoramica sulla formazione che ho ricevuto, sia quella in ambito teorico-economico sia quella in ambito tecnico.

\subsection{Conoscenze economiche acquisite}
In questa sottosezione parlerò più approfonditamente della formazione in ambito teorico-economico che ho ricevuto e riporterò i macro argomenti trattati, magari con qualche accenno a qualche concetto fondamentale.

\subsubsection{ELISE}
In questa sotto sottosezione descriverò ELISE e spiegherò i suoi scopi e funzionalità principali, collegando infine il tutto con le attività formazione teorico-economiche che ho applicato in essa dopo averle acquisite.

\subsection{Tecnologie utilizzate}
In questa sottosezione parlerò più dettagliatamente delle tecnologie che ho utilizzato.

\subsubsection{Piattaforma host}
In questa sotto sottosezione introdurrò la piattaforma host su cui ho lavorato, spiegando nel dettaglio la struttura della rete e i vari elaboratori su cui il codice viene immagazzinato ed eseguito.
	
\subsubsection{COBOL}
In questa sotto sottosezione riporterò i concetti fondamentali di quello che è il linguaggio di programmazione COBOL e l'uso che se ne fa, riportando poi alcuni esempio dei formalismi principali tramite i quali i moduli delle applicazioni host elaborano i dati.
	
\subsubsection{JCL}
In questa sotto sottosezione riporterò i concetti fondamentali di quello che è il linguaggio di scripting JCL e riportando qualche esempio dell'uso che se ne fa, ovvero il controllo dell'esecuzione di moduli COBOL.

\section{Processo di sviluppo}
In questo inizio di sezione introdurrò la fase di stage relativa al processo di sviluppo del progetto di stage.

\subsection{Analisi dei requisiti}
In questa sottosezione descriverò l'attività di analisi dei requisiti e riporterò il modo in cui è stata svolta.

\subsection{Progettazione}
In questa sottosezione descriverò l'attività di progettazione delle implementazioni richieste per lo svolgimento del progetto di stage.
		
\subsection{Codifica}
In questa sottosezione descriverò l'attività di codifica delle implementazioni richieste per lo svolgimento del progetto di stage.
	
\section{Verifica e validazione}
In questa sezione descriverò le attività di Verifica e Validazione delle implementazioni richieste per lo svolgimento del progetto di stage, in particolare descriverò singolarmente l'analisi statica e dinamica effettuata a questo fine.

\subsection{Analisi statica}
In questa sottosezione parlerò degli strumenti utilizzati per l'analisi statica del codice prodotto.
	
\subsection{Analisi dinamica}
In questa sottosezione parlerò dei metodi utilizzati per l'analisi dinamica che utilizza la divisione.

\subsection{Collaudo}
In questa sottosezione parlerò dei metodi utilizzati per il collaudo che utilizza la divisione. Nello specifico parlerò della figura di collaudatore e dei documenti di collaudo che l'azienda proponente richiede ad ogni rilascio.
	
\section{Valutazione del prodotto}
In questa sezione parlerò della valutazione dei documenti e del prodotto dopo i vari test e collaudi.